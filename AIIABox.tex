\Org{Artificial Intelligence and Intelligence Augmentation}{
The driving question of AI can be summarized as: ``are machines
  capable of behaving in a way that indistinguishable from that of
  humans, as judged by other humans''.  The archetypal test
  of whether artificial intelligence has been achieved is the {\em Turing
  Test}, in which a human, communicating with another agent through
  text alone, is unable to tell whether or not the agent is human. A
  natural consequence of computers being indistinguishable from humans
  is that they will be replacing humans, causing mass unemployment.

  The driving question of IA is whether and how computers can be used 
to {\em augment} humans rather than replace them. Some augmentations
are the territory of science fiction. For example,
cyborgs whose anatomy is part human, part artificial and can with
equal ease solve complex equations or write poetry. Other examples are
so mundane that they are taken for granted. Examples
are the smart phone and google search, with which our capabilities are augmented by computers. 
 
The Turing test was published~\cite{turing1951can} in 1951. A 1956
workshop in Dartmouth college is widely recognized as the beginning of
the field of AI. IA appeared on the scene soon thereafter, with
Ashby~\cite{ashby1957introduction} in 1957
Licklider~\cite{licklider1960man} in 1960 and
Englbart~\cite{engelbart1962augmenting} in 1962.

Arguably, the impact of IA on today's society is much larger than
that of AI. Siri, Google search and assisted driving are some of the
common apps that augment human ability. On the other hand, the goal of
creating a general purpose AI that possesses a human-like capability
to reason about new domains seems to be as far as ever.  At the same
time, AI holds the fascination of many, possibly because of its
tantalizing combination of promise and threat.}

