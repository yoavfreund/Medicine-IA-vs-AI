\Medicine{Alarm fatigue}{ Patient monitors are bedside medical devices
  that monitor patients that are at risk but currently stable, freeing
  the medical staff to attend to the patients whose status is
  critical.  Unfortunately, Patient monitors suffer from signal
  quality issues and tend to generate false alarms at a high
  rate. Over time, this can result in the staff not responding to
  alarms, potentially resulting in great damage to the patient. This
  phenomenon, called {\em alarm fatigue} (or alarm overload) is a
  major problem in hospital care \cite{cvach2012monitor,brief2019top}.
  
  \yoav{Two questions: (1) Which of these papers are about the severity of the problem and which are about research? (2) regarding research, has there been research aimed at making the monitors adaptive to the patient and or nurse?}
  
  Alarm fatigue is a well known issue medical providers encounter when working with patient monitors. It is frequently named as a threat to patient safety \cite{sendelbach2013alarm,ruskin2015alarm}, and a lot of research has been carried out toward this problem \cite{cvach2012monitor,paine2016systematic,bai2016sequence,hu2019algorithm}.
}
