\Medicine{Alarm fatigue}{ Patient monitors are bedside medical devices
  that monitor patients that are at risk but currently stable, freeing
  the medical staff to attend to the patients whose status is
  critical.  Unfortunately, Patient monitors suffer from signal
  quality issues and tend to generate false alarms at a high
  rate. Over time, this can result in the staff not responding to
  alarms, potentially resulting in great damage to the patient. This
  phenomenon, called {\em alarm fatigue} (or alarm overload) is a
  major problem in hospital care \cite{brief2019top}. See, for example,  \cite{cvach2012monitor,paine2016systematic}, for a review.
  
  Alarm fatigue is a well known issue medical staff and is considered a threat to patient safety \cite{sendelbach2013alarm,ruskin2015alarm}. There is plenty of published research on reducing false alarm rates using different techniques such as signal quality control \cite{li2012signal,behar2013ecg} and superalarm \cite{bai2016sequence,hu2019algorithm}.
}
