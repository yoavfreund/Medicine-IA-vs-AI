   \Medicine{knowledge gap}{ 
  
  
  
  {\color{blue} At the writing of this article the COVID-19 is a worldwide crisis \cite{sohrabi2020world}. Back in Jan 2020, when it was first reported in China, nobody had a clue how it will generate damage to human body, not to mention how to treat a patient. All medical practices, ranging from diagnosis to treatment to vaccine were all made based on experience. For example, the quinine and remdesivir were considered potential for hospitalized patients. A lot of data could be collected, but there is a huge gap between the data and what's going on. Knowledge was quickly accumulated in the past few months. For example, we know more about hydroxychloroquine, remdesivir, and many other candidate drugs; see, for example \cite{sanders2020pharmacologic,goldman2020remdesivir}. However, due to the lack of knowledge, even if there exist some protocols, for example \cite{world2020population,world2020protocol,nakajima2020covid,awad2020perioperative} and several others, there are still many white and unknown details in the overall medical practice.} 
  
  Knowledge gap is not confined to newly emerging diseases. Urodynamics can trace its root back to the 1800s \cite{perez1992history}, and recently become exponentially important in clinics. 
  %
  Urodynamic studies provide the best bladder and sphincter functional data for urologists to decide how to treat patients at risk for renal damage \cite{abrams2003describing}, or the diagnostic information for the cause and nature of a patient's incontinence, frequent urination, recurrent urinary tract infections, etc. 
  %
  While it has been extensively studied and applied in clinics, the main issue that plagues this field of urodynamics is the lack of precise definition of a detrusor contraction or overactive contraction \cite{abrams2003describing}.
  %
  For example, the protocols available today for reading the time series corrected from urodynamics, like International Children's Continence Society and International Continence Society, are mixed up by descriptive and quantitative statements \cite{bauer2015international,drake2018fundamentals}. 
%  
  The lack of a well defined definition is due to the lack of quantitative study from the pathophysiological perspective. For example, usually an overactive contraction represents itself as a ``bump'' in the detrusor pressure signal. However, what is the breath and height, or the shape, of a bump should we call it an overactive contraction? How to distinguish a true overactive contraction from an artifact? While there have been several reference information, like abdominal pressure, that could help us identify artifacts, but it can only explain a small portion of them. 
%
  Unsurprisingly, this fundamental issue has led to a significant
  inter-rater disagreement
  \cite{venhola2003interobserver,dudley2018interrater}.
  }

