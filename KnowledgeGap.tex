   \Medicine{Knowledge gap}{ 
  Medical science is constantly evolving. This means
  that at any point of time, some medical facts are outside the
  collective knowledge of the medical profession. We refer to this as
  the {\em knowledge gap}.
  
  As the writing of this article {\em COVID-19} is a worldwide crisis
  \cite{sohrabi2020world}. Back in Jan 2020, when it was first
  reported in China, very little was known about the disease or how to
  treat it.  Knowledge was quickly accumulated during the last 
  months. For example, we know more about hydroxychloroquine,
  remdesivir, and other candidate
  drugs~\cite{sanders2020pharmacologic,goldman2020remdesivir} and some
  treatment protocols have been
  developed~\cite{world2020population,world2020protocol,nakajima2020covid,awad2020perioperative}.
  Still, much is still unknown about COVID-19.
  
Significant knowledge gaps exist for some {\em well known maladies}.  For
example, Urodynamics, whose aim is to understand the movement of urine
through the bladder, sphincters, and urethra has been studied since
the 1800's~\cite{perez1992history}.
Urodynamic studies provide reliable time series of the pressure
dynamics in the bladder and sphincter. These time series are used
by urologists to decide how to treat patients at risk for
renal damage~\cite{abrams2003describing}, incontinence, frequent
urination, recurrent urinary tract infections, etc.
There are several well-established protocols for interpreting
detrusor pressure time
series.\cite{bauer2015international,austin2016standardization}.
When following these protocols, the Eurologist needs to identify
occurances of an important short-term event called {\em detrusor
overactivity}.  However, as there is no precise definition of a detrusor
overactivity, the identification of these events is based 
on subjective judgements.  This has led to a significant inter-rater
disagreement in the analysis of detrusor pressure time series.
\cite{venhola2003interobserver,dudley2018interrater}.
}

