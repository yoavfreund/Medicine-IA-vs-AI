   \Medicine{Knowledge gap}{ 
  Medical science is constantly evolving. On the flip side, this means
  that at any point of time, some medical facts are outside the
  collective knowledge of the medical profession. We refer to this as
  the {\em knowledge gap}.
  
  As the writing of this article the COVID-19 is a worldwide crisis
  \cite{sohrabi2020world}. Back in Jan 2020, when it was first
  reported in China, very little was known about the disease or how to
  treat it.  Knowledge was quickly accumulated during the last 
  months. For example, we know more about hydroxychloroquine,
  remdesivir, and other candidate
  drugs~\cite{sanders2020pharmacologic,goldman2020remdesivir} and some
  treatment protocols have been
  developed~\cite{world2020population,world2020protocol,nakajima2020covid,awad2020perioperative}.
  Still, much is still unknown about COVID-19.
  

  Significant knowledge gaps exist for some diseases that have been known
  for a long time. For example, Urodynamics,  whose aim is to
  understand the movement of urine through the bladder, sphincters,
  and urethra has been studied since the
  1800's~\cite{perez1992history}.



  Urodynamic studies %time series \yoav{Maybe give a sentence about what is measured? Pressure in the bladder?} 
  provide reliable 
  bladder and sphincter information, which is %function data 
  used by urologists to decide how to treat patients at risk for
  renal damage~\cite{abrams2003describing}, incontinence,
  frequent urination, recurrent urinary tract infections, etc.
  %Urodynamic studies are extensively studied and applied. One of the
  %main issues that plagues this field is the lack of
  %precise definition of a detrusor contraction or overactive
  %contraction \cite{abrams2003describing}. \yoav{Does the doctor need
  %to distinguish between these two?}
  %
%\yoav{ Needs more focus: is the central issue the difference between
%detruser contraction and overactive contraction? If so, lets focus on that.}
  % 
  For example, the protocols available today for reading the time
  series collected from urodynamics, like International Children's
  Continence Society and International Continence Society, are a combination of
  descriptive and quantitative statements
  \cite{bauer2015international,austin2016standardization}.
Normal voiding happens when the bladder outlet, controlled by a sphincter, relaxes and the detrusor, a smooth muscle found in the wall of the bladder,  contracts. The detrusor muscle activity, or detrusor pressure time series, is calculated by subtracting the abdominal pressure measured with a rectal catheter from the vesical pressure measured with a catheter in the bladder.
A critical element in the analysis of a detrusor pressure time series is the {\em detrusor overactivity}.
However, there is no precise definition of an detrusor overactivity, so the identification of detrusor overactivity is based on the doctors subjective judgement.
  Unsurprisingly, this fundamental issue has led to a significant
  inter-rater disagreement
  \cite{venhola2003interobserver,dudley2018interrater}.
  }

