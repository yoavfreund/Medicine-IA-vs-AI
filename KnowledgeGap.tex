   \Medicine{Knowledge gap}{ 
  Medical science is constantly evolving. On the flip side, this means
  that at any point of time, some medical facts are outside the
  collective knowledge of the medical profession. We refer to this as
  the {\em knowledge gap}.
  
  As the writing of this article the COVID-19 is a worldwide crisis
  \cite{sohrabi2020world}. Back in Jan 2020, when it was first
  reported in China, very little was known about the disease or how to
  treat it.  Knowledge was quickly accumulated during the last 
  months. For example, we know more about hydroxychloroquine,
  remdesivir, and other candidate
  drugs~\cite{sanders2020pharmacologic,goldman2020remdesivir} and some
  treatment protocols have been
  developed~\cite{world2020population,world2020protocol,nakajima2020covid,awad2020perioperative}.
  Still, there are still many white and unknown details in the treatment of
  COVID-19.
  \yoav{What is a ``white detail''?}

  Significant knowledge gaps exist for some diseases that have been known
  for a long time. For example, Urodynamics,  whose aim is to
  understand the movement of urine through the bladder, sphincters,
  and urethra has been studied since the
  1800's~\cite{perez1992history}.

  Urodynamic studies time series \yoav{Maybe give a sentence
  about what is measured? Pressure in the bladder?} provide reliable bladder and sphincter functional
  data used by urologists to decide how to treat patients at risk for
  renal damage~\cite{abrams2003describing}, incontinence,
  frequent urination, recurrent urinary tract infections, etc.
  
  Urodynamic studies are extensively studied and applied. One of the
  main issues that plagues this field is the lack of
  precise definition of a detrusor contraction or overactive
  contraction \cite{abrams2003describing}. \yoav{Does the doctor need
  to distinguish between these two?}
  %

\yoav{ Needs more focus: is the central issue the difference between
detruser contraction and overactive contraction? If so, lets focus on that.}

  % 
  For example, the protocols available today for reading the time
  series corrected from urodynamics, like International Children's
  Continence Society and International Continence Society, are mixed
  up by descriptive and quantitative statements
  \cite{bauer2015international,drake2018fundamentals}.
%  
  The lack of a well defined definition is due to the lack of
  quantitative study from the pathophysiological perspective. For
  example, usually an overactive contraction represents itself as a
  ``bump'' in the detrusor pressure signal. However, what is the
  breath and height, or the shape, of a bump should we call it an
  overactive contraction? How to distinguish a true overactive
  contraction from an artifact? While there have been several
  reference information, like abdominal pressure, that could help us
  identify artifacts, but it can only explain a small portion of them.
%
  Unsurprisingly, this fundamental issue has led to a significant
  inter-rater disagreement
  \cite{venhola2003interobserver,dudley2018interrater}.
  }

