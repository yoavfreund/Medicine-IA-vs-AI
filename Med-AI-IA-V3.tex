\documentclass[11pt]{pnas-new}
% \documentclass[10pt]{article}
\templatetype{pnasresearcharticle} % Choose template 
% {pnasresearcharticle} = Template for a two-column research article
% {pnasmathematics} %= Template for a one-column mathematics article
% {pnasinvited} %= Template for a PNAS invited submission

\usepackage[utf8]{inputenc}
\usepackage[T1]{fontenc}
%\usepackage{xcolor,ulem}
\usepackage{mdframed}
\usepackage{wrapfig}
\usepackage{multicol}
\setlength{\columnsep}{1cm}
\usepackage{graphicx}
\usepackage{adjustbox}
\usepackage{lipsum}
\newlength{\strutheight}
%\usepackage{soul} % for strike-through (\st)

\author[1]{Hau-Tieng Wu}
\author[2]{Yoav Freund}

\affil[1]{Duke, Mathematics and Statistical Science, Durham, 27708, USA}
\affil[2]{UCSD, Computer Science, San Diego, 92093, United States. yfreund@eng.ucsd.edu}


\title{Trusted digital doctors know their limits}

\newcommand{\comment}[3]{{\color{#1} {\bf #2 :} #3}}
%\newcommand{\comment}[3]{}  % suppress comments
\newcommand{\hautieng}[1]{\comment{blue}{Hautieng}{#1}}
\newcommand{\yoav}[1]{\comment{red}{Yoav}{#1}}

\mdfsetup{middlelinecolor=blue, middlelinewidth=2pt, backgroundcolor=blue!10, roundcorner=10pt}

\mdfdefinestyle{Medicine}{middlelinecolor=red, middlelinewidth=2pt, backgroundcolor=red!10, roundcorner=10pt}
\mdfdefinestyle{ML}{middlelinecolor=green, middlelinewidth=2pt, backgroundcolor=green!10, roundcorner=10pt}
\mdfdefinestyle{Org}{middlelinecolor=blue, middlelinewidth=2pt, backgroundcolor=blue!10, roundcorner=10pt}

\newcommand{\block}[3]{
  \begin{wrapfigure}{r}[34pt]{-10pt}
    \begin{minipage}[t]{12cm}
      \begin{mdframed}[style=#1]{\footnotesize{\bf #2}\\ #3} \end{mdframed}
    \end{minipage}
  \end{wrapfigure}
}
\newcommand{\Medicine}[2]{\block{Medicine}{#1}{#2}}
\newcommand{\ML}[2]{\block{ML}{#1}{#2}}
\newcommand{\Org}[2]{\block{Org}{#1}{#2}}


\renewcommand{\input}[1]{}

\begin{abstract}

  The meteoric rise of AI in general and Deep Learning in particular
  is generating great excitement throughout academia and commerce, and
  in particular in medicine\cite{topol2019deep,
    wachter2015digital}. With some some high-profile claims
  that AI will soon replace humans in many medical specialties.

  In this position paper we present an alternative view. We contrast
  {\em Artificial Intelligence} with {\em Intelligence Augmentation}
  and argue that the second is more likely to benefit the patient than
  the first. We provide evidence to this argument and present a vision
  in which easier decisions are delegated to computers, while the more
  difficult ones are handled by humans.

\end{abstract}

\begin{document}
\settoheight{\strutheight}{\strut}

 
\maketitle

%\thispagestyle{firststyle}
\iffalse
\section{issues to be resolved}

\hautieng{As we discussed, I followed the medical journal convention and changed the author order so that you are the senior/corresponding author.}

\begin{itemize}
    \item Introduce the term "Healthcare Provider" with acronym HP, and use throughout instead of doctors nurses etc. \hautieng{I am not sure what is the best way to do it. In many places, "doctor" are clearly more precise than HP. So I leave this part to you.}
    \item Patient compliance - how can IA help make sure that patients take their meds, don’t drink excessively etc. \hautieng{done}
    \item Compensation: Doctors that generate quality data should own this data, distribution mechanisms should compensate the doctor for his/her contribution. \hautieng{As we discussed, this might be a bit off the topic. This is more like a regulatory issue. Let's discuss it if you have a different viewpoint.}
    \item A few grammatical corrections:
    -insert 2 page 2 last line “piopsies” instead of biopsies
    -last line page 2 spelling: “therefor” (needs an e) therefore 
    -insert page 7 we quote from Robert Rechter’s book... (53) Reference @ 53 indicates “Robert Wachter” as author. \hautieng{I am not fully sure what you want...}

\end{itemize}

To Highlight a change use this macro: \change{old text}{new text}
\fi

\section{Introduction}

The meteoric rise of AI and Deep learning raises the possibility that
doctors will be replaced by computers~\cite{Mukherjee2017}. Geoff Hinton,
a famous deep learning researcher said in 2017: ``It's just completely
obvious that in ten years deep learning is going to do better than
Radiologists ... They should stop training radiologists now''.

The predictions of Sebastian
Thrun~\cite{Mukherjee2017,esteva2017dermatologist}, another leader in
machine learning, are less disruptive: ``... deep learning devices
will not replace dermatologists and radiologists. They will {\em
  augment} professionals, offering the expertise and assistance''. 
  The term "Augmented Intelligence" appears frequently in recent 
  policy reports from medical associations~\cite{american2019augmented,AAD2019augmented}. Collaborations between computers and human care-givers are seen as a desirable goal.
  In this article we suggest how such a collaboration between human and computer might unfold in practice. Central to our approach is to allow the computer to quantify it's own uncertainty and, when appropriate, output "I don't know" (which we will abbreviate as IDK).
 
\Org{Artificial Intelligence and Intelligence Augmentation}{
The driving question of AI can be summarized as: ``are machines
  capable of behaving in a way that indistinguishable from that of
  humans, as judged by other humans''.  The archetypal test
  of whether artificial intelligence has been achieved is the {\em Turing
  Test}, in which a human, communicating with another agent through
  text alone, is unable to tell whether or not the agent is human. A
  natural consequence of computers being indistinguishable from humans
  is that they will be replacing humans, causing mass unemployment.

  The driving question of IA is whether and how computers can be used 
to {\em augment} humans rather than replace them. Some augmentations
are the territory of science fiction. For example,
cyborgs whose anatomy is part human, part artificial and can with
equal ease solve complex equations or write poetry. Other examples are
so mundane that they are taken for granted. Examples
are the smart phone and google search, with which our capabilities are augmented by computers. 
 
The Turing test was published~\cite{turing1951can} in 1951. A 1956
workshop in Dartmouth college is widely recognized as the beginning of
the field of AI. IA appeared on the scene around the same time, with
Ashby~\cite{ashby1956introduction} in 1956
Licklider~\cite{licklider1960man} in 1960 and
Englbart~\cite{engelbart1962augmenting} in 1962.

Arguably, the impact of IA on today's society is much larger than
that of AI. Siri, Google search and assisted driving are some of the
common apps that augment human ability. On the other hand, the goal of
creating a general purpose AI that possesses a human-like capability
to reason about new domains seems to be as far as ever.  At the same
time, AI holds the fascination of many, possibly because of its
tantalizing combination of promise and threat.}

 The argument as to whether human healthcare providers (HP~\footnote{Our discussion applies to all healthcare providers, including Doctors, Nurse Practitioners and Nurses. To simplify the references o this group of professionals we use the acronym HP, which stands for "Healthcare Provider"}  will be replaced or empowered by computers is a recent
incarnation of a long standing debate between AI (Artificial Intelligence) and IA
(Intelligence amplification) (see inset).

To distinguish between the goals of AI and IA we use the 
terms ``AI agent'' and ``IA
sidekick''. This terminology contrasts {\em agents}, which are endowed
with {\em agency} and can take {\em actions} that effect the patient's
health, with {\em sidekicks}, which can provide advice and suggestions,
but who are not allowed to take an action.

Of the many potential uses of AI/IA in medicine, we focus on diagnostics.
AI-driven diagnosis are expected to become a reality much sooner than other medical activities such as surgery~\cite{topol2019deep}. 

Central to our argument is a quantification of {\em prediction
  confidence}. Such quantification is needed to avoid premature
diagnostic conclusions, and to decide which additional tests or
consultations might be needed. Consider a doctor that is asked
to diagnose a patient with complex or conflicting symptoms. A careful
doctor will admit their uncertainty and perform additional tests or
ask a specialist. A less careful, overly self confident doctor is
likely give an incorrect diagnosis and choose an ineffective or even damaging
treatment plan.

An AI agent, expected to be better than the human HP, is more likely to 
end up behaving like an overly confident doctor. An IA sidekick, aware of
it's own limitations, will give advice only when the evidence is
strong and otherwise say ``I don't know''.

In the following sections we explore these ideas in more detail. We
start with a critique of one of the papers that claims that AI agents
can outperform human diagnosticians.

\section{Supervised Learning and the Ground Truth}
\label{sec:ground-truth}

Deep learning (DNN) is a special case of {\em supervised learning} (see
inset), sometimes called {\em input-output}
learning~\cite{ng2016artificial,topol2019deep}.

Examples of applications of DNN to medical diagnosis include arrhythmia classification 
from single channel electrocardiogram \cite{hannun2019cardiologist}, diabetic retinopathy detection from retinal fundus photographs \cite{gulshan2016development}, skin cancer diagnostics~\cite{esteva2017dermatologist} as well as many others.

\ML{Supervised Learning and ground truth}{Roughly speaking, machine
  learning (ML) can be divided into {\em unsupervised} learning and
  {\em supervised} learning. In both cases, the task of the learning
  algorithm is transforming a set of {\em examples} into a {\em model}. In
  unsupervised learning, the examples are undifferentiated raw
  measurements. In supervised learning, which is the focus of
  this article, each example consists of an {\em input} and a {\em
    label}. Typically, the labels are provided by human
  experts. These labels {\color{red}are usually viewed as}\sout{define} the {\em ground truth} and the goal of
  the learning algorithm is to make predictions that diverge as little
  as possible from the ground truth.}


The data for supervised learning consists of a large collection of
(input, output) pairs. For medical diagnosis, usually the input is medical
information for the patient (Heart rate, blood tests, X-ray images
etc.) and the output is the diagnosis. This output is considered the
``ground-truth'' and is assumed to represent the undisputed truth.

Here lies the first difficulty with applying supervised learning
to medical diagnosis. In this context the labels correspond to 
the judgement of human diagnosticians. However 
different diagnosticians might give different diagnoses, in which case it is hard to assign ground truth. 
We will return to this important issue in the next section.

The other important assumption made in supervised learning is that the
generated classifier is tested using the same distribution of examples
as that of the training set.

\ML{Skin cancer diagnosis using Deep Neural Networks}{One of the
  papers that provided evidence that deep neural networks might be
  able to outperform humans is the work of Esteva et
  al~\cite{esteva2017dermatologist}. They trained a Deep neural
  network to classify images of skin into three categories: benign,
  malignant and non-cancerous. The network was then tested, along with
  twenty five dermatologists on images which were labeled by a
  pathologist analysis of the biopsy. The neural network performed
  comparably to, and sometimes better than the human dermatologist.
  To provide ground truth, the patients were biopsied and the piopsies
  were diagnosed by pathologists.
}  

A highly cited paper in the journal Science~\cite{esteva2017dermatologist}
makes the claim that DNNs can perform diagnostics as well as, or better, than board certified dermatologists. (see inset)

A fundamental problem with the experiments presented in the paper is 
the way the data was
collected. The data used in the experiment was {\em retrospective},
i.e., it was collected from the records of past patients for which both
a skin image and a biopsy were available. Normally, patients get
biopsied only if the dermatologist thinks there is a significant
chance of {\bf malignancy}. As a result, a retrospective study that is
based on patients for whom a biopsy was taken is likely to
over-represent malignant patients and therefore be biased. If an image-based classifier
is trained on the biased data, its performance on unbiased test data
is likely to be worse. Specifically, when the classifier is applied to skin
images of un-diagnosed patients it is likely to over-diagnose them as
malignant. The practical implication would be that more patients than
necessary will be biopsied. 

As we elaborate on in the next section, in medical diagnostics the
{\em ground truth} are often unavailable, all that we have to go on are
the opinions of human diagnosticians.

\section{Uncertainty in medicine}

\input{Arrhythmia}
One of the methods of quantifying uncertainty in medical diagnosis is inter-rater studies.
In studies of this kind, multiple doctors produce diagnostics
based on identical medical information without communicating with each other. (see inset).
Associating {\em ground Truth} with these diagnostics whose inter-rater agreement is less than perfect is challenging.
 
In addition, diagnosis is not an
input-output mapping. Rather, it is an iterative process which reduces
uncertainty over time. To illustrate this, consider the diagnostics of
a patient that is treated in an out-patient clinic.  When a patient
arrives at a clinic for the first time, all diagnostics are
possible. After a physical exam and an interview with a doctor, many
possibilities are eliminated.
\Medicine{Inter-rater agreement}{A common method for measuring the
  level of agreement is an {\em inter-rater agreement studies}. In
  such studies several doctors are provided with the same patient
  file and are asked to give a diagnosis.

  A common measure of of the agreement between two raters is the Cohen's kappa
  coefficient, usually denoted by $\kappa$.  Kappa is computed from two more basic quantities: $0\leq a\leq 1$
  is the fraction of patient files on which the two raters agree, 
  and $0\leq c\leq 1$ is the fraction of agreements that would occur
  by chance. The definition of kappa is $\kappa=\frac{a-c}{1-c}$.

  If $\kappa=1$ The raters always agree, if $\kappa=0$ the rate of
  agreement corresponds to chance, and if $\kappa<0$ then the rate of
  agreement is lower than chance, i.e. the two raters tend to have
  different opinion. An
    interpretation of $\kappa$ recommended by Cohen
    \cite{mchugh2012interrater} is: $\kappa\leq 0$: no agreement, $0< \kappa\leq 0.20$:none to slight
    agreement, $0.2<\kappa\leq 0.40$: fair agreement,
    $0.4<\kappa\leq 0.60$ as moderate agreement, $0.6<\kappa\leq
    0.80$: substantial agreement, and $0.8<\kappa\leq 1.00$: perfect
    agreement.

    For example, in the sleep stage annotation, the pairwise Cohen's kappa over 5 sleep experts (totally 10 pairs) is on average 65\% over normal subjects and about 59\% over subjects with sleep apnea \cite{norman2000interobserver}. In other words, the inter-rater agreement is substantial over normal subjects and moderate over subjects with sleep apnea.
    %\yoav{can we give a list of kappa values for some common      diagnostics here?}
  }


In {\em simple} cases, this is enough
for the doctor to confidently choose a treatment. In more complex
cases, the doctor might ask for multiple tests and visits, refer the
patient to a specialist, consult colleagues, journals and books,
etc. To choose a treatment plan, the set of possible diagnostics has
to be reduced.  However, it does not have to be reduced to a {\em
  single} diagnosis before treatment can be started.

In order to apply a supervised learning method, such as  DNN, to the
diagnostics problem, one needs to define a {\em ground-truth} diagnostic for each
patient. However, that is easier said than done. Currently, most of
the data available for machine learning is {\em observational or
retrospective data}; that is, data has been collected
during the diagnosis and treatment of patients.
One way to confirm a diagnosis is to collect more information through
additional examinations and tests. However the attending medic might
judge additional tests to be unnecessary, in which case the outcome of these tests will not
appear in the observational study.
Another way is following up on the treatment plan. However, correct
diagnosis is only one of many factors that influence the long term
health of the patient. Others include choice of treatment, medication
adherence, change in diet or reduction in stress, among many others.

\Medicine{Uncertainty due to signal quality}{Medical devices use a
variety of bio-sensors that measure record and analyze different
biometric signals. These signals vary in velocity, ranging from
low-velocity vital signs such as heart rate, oxygen saturation,
temperature and blood pressure, through high-velocity waveforms such
as ECG and EEG and medical imaging such as CT, X-ray, MRI and scanning
microscopes. Some signals, such as blood pressure or heart rate, can
be directly used in diagnosis.  Some other signals have to be
interpreted before they can be used in diagnosis. We use the acronym
HS (High Speed) to refer to those signals that require interpretation.

Interpreting HS is a significant fraction of the work of most
doctors. In addition, There are medical specialties such as
Radiology and Pathology that are devoted to interpreting HS. These
so-called ``Pattern Doctors'' are predicted to be the early adopters
of AI~\cite{topol2019deep} or IA.
HS provides critical detailed information about the patient's
health. However, the richness of the signal can make it susceptible to
nuisance variability from noise, limited resolution, operator error,
etc. The number of nuisance variables is very large and confounding. In the next
paragraph we give an example of one nuisance variable: the placement
of ECG leads on the patient's body.
  
For ECG signals to be correctly and consistently interpreted, it is
important that the leads are placed correctly on the patient's
body. Several standards for placement have been published, for example, the standard 12 leads ECG system
\cite{goldberger2017clinical}, the EASI system \cite{Dower1988}, and the Frank lead system \cite{frank1956accurate}. 
It is not always possible to achieve a consistent and precise sensor
placement for biomedical signal collection due to various reasons, for example, the torso variation caused by gender, age and living styles.
This uncertainty might be tolerable for some clinical applications;
for example, an imprecise ECG sensor placement might not impact the
identification of some types of arrhythmia from the ECG signal, like
atrial fibrillation. However, this uncertainty might cause troubles in
identifying other types of arrhythmia, for example, premature atrial contraction.

}

There are many causes for uncertainty in medical diagnosis. We briefly
describe three categories of problems: {\em signal quality}, the {\em
  knowledge gap} and the limitations of {\em diagnostic protocols}.

By {\em Signal Quality} we refer to the quality of the raw data
collected for medical diagnosis. Some diagnostic measures, such
as heart rate, blood pressure and temperature can be measured more reliably
and consistently than other
measures such as EKG, EEG, camera images, X-ray, ultrasound and MRI, some of which might
produce vast and highly variable data. The quality of this data
depends on many factors among them, including the quality of the instruments, the
consistency of the human operator, the build of the patient, etc.

\Medicine{Alarm fatigue}{ Patient monitors are bedside medical devices
  that monitor patients that are at risk but currently stable, freeing
  the medical staff to attend to the patients whose status is
  critical.  Unfortunately, Patient monitors suffer from signal
  quality issues and tend to generate false alarms at a high
  rate. Over time, this can result in the staff not responding to
  alarms, potentially resulting in great damage to the patient. This
  phenomenon, called {\em alarm fatigue} (or alarm overload) is a
  major problem in hospital care \cite{brief2019top}. See, for example,  \cite{cvach2012monitor,paine2016systematic}, for a review.
  
  \yoav{Two questions: (1) Which of these papers are about the severity of the problem and which are about research? {\color{blue}-- Updated.} (2) regarding research, has there been research aimed at making the monitors adaptive to the patient and or nurse? {\color{blue}-- to my knowledge, they are adaptive to patients. The focus is reducing the patients' risk by reducing the false alarm.}} 
  
  Alarm fatigue is a well known issue medical providers encounter when working with patient monitors. It is frequently named as a threat to patient safety \cite{sendelbach2013alarm,ruskin2015alarm}, and a lot of research has been carried out toward this problem based on different techniques; for example, signal quality control \cite{li2012signal,behar2013ecg} or superalarm \cite{bai2016sequence,hu2019algorithm}, to name but a few.
}

Signal quality enhancement is already an important part of imaging
devices such as X-ray and MRI. Methods such as compressed
sensing~\cite{lustig2008compressed} are used to reconstruct 3d images
from a large number of noisy scans.

One situation where signal quality and signal variability cannot be
neglected is Patient Monitors.  The purpose of these devices is to
continuously monitor patients vital signs and alert the medical staff
if a dangerous situation is detected. Unfortunately, the false alarm
rate of these devices is often high. This results in a phenomenon
called ``alarm fatigue'' (see inset) where the medical staff ignores
the generated alarms, rendering them less clinically useful than
expected.

Signal quality and alarm fatigue can be thought of as ``bottom up''
causes of uncertainty. The uncertainty originates in the medical
devices and moves up to the HP.

Other types of uncertainty are ``top down'' in that they originate in medical research percolates down to the HP. We briefly
describe two types of top-down uncertainty: knowledge gaps and the
limitation of medical protocols.

{\em ``Knowledge gap''} corresponds to limitations of scientific medical
knowledge. This is not the limitation of a particular doctor; rather,
it reflects the limitations of knowledge that correspond to successful
medical trials.
   \Medicine{Knowledge gap}{ 
  Medical science is constantly evolving. This means
  that at any point of time, some medical facts are outside the
  collective knowledge of the medical profession. We refer to this as
  the {\em knowledge gap}.
  
  As the writing of this article, {\em COVID-19} is a worldwide crisis
  \cite{sohrabi2020world}. Back in Jan 2020, when it was first
  reported in China, very little was known about the disease or how to
  treat it.  Knowledge was quickly accumulated during the last 
  months. For example, we know more about hydroxychloroquine,
  remdesivir, and other candidate
  drugs~\cite{sanders2020pharmacologic,goldman2020remdesivir} and some
  treatment protocols have been
  developed~\cite{world2020population,world2020protocol,nakajima2020covid,awad2020perioperative}.
  Still, much is still unknown about COVID-19.
  
Knowledge gaps not only exist in new diseases, but
also exist in some well known maladies that have been studied for a
long time.
For example, Urodynamics, whose aim is to understand the movement of urine
through the bladder, sphincters, and urethra has been studied since
the 1800's~\cite{perez1992history}.
Urodynamic studies provide reliable time series of the pressure
dynamics in the bladder and sphincter. These time series are used
by urologists to decide how to treat patients at risk for
renal damage~\cite{abrams2003describing}, incontinence, frequent
urination, recurrent urinary tract infections, etc.
There are several well-established protocols for interpreting
detrusor pressure time
series.\cite{bauer2015international,austin2016standardization}.
When following these protocols, the urologist needs to identify
occurrences of an important short-term event called {\em detrusor
overactivity}.  However, as there is no precise definition of a detrusor
overactivity, the identification of these events is based 
on subjective judgements.  This has led to a significant inter-rater
disagreement in the analysis of detrusor pressure time series.
\cite{venhola2003interobserver,dudley2018interrater}.
}



Even when medical knowledge exists, an individual doctor might not
possess it. The dissemination of up to date and reliable medical
information is far from even. One of the important information
dissemination tools are {\em medical protocols}. Those are used to
ensure uniformity and consistency of treatment between hospitals,
doctors and nurses. While protocols are an important dissemination
tool, they have some limitations, as described in the inset.
~\\

\section{Uncertainty in machine learning}

We described some of the many causes of uncertainty in medical
diagnosis. One might conclude that the automation of diagnosis in
diseases with high rates of inter-rater disagreement is
impossible. However, this is not necessarily the case. There are types
of diagnostics for which the typical case is {\em simple}, and only a
small fraction of the cases are {\em complex}.  We use the term
``simple'' to mean that {\em most doctors are likely to give the same
  diagnosis}.  In other words, these are the cases on which
inter-rater {\em dis}agreement is low. ``Complex'' cases are those where
doctors tend to either say "I don't know" or disagree with each other.
{\em Our proposal is to allow IA to output IDK on
  the complex cases, and, in exchange, require higher levels of
  accuracy and reliability when the IA makes a prediction.}

Increasing confidence in a diagnosis by seeking consensus among
several doctors is common sense. A similar approach has been used in
machine learning algorithms such as Bagging~\cite{breiman1996bagging},
Random Forests~\cite{breiman2001random}, and
Boosting\cite{SchapireFr2012}. These so-called ``ensemble'' algorithms
take the majority vote of predictions from several ``base'' learning
algorithms using a majority vote to generate a single more reliable
prediction.~\footnote{A majority is used when there are only two
  possible labels. A more general combination rule is the {\em
    plurality} i.e., the label that gets the largest number of votes.}

The majority vote always outputs one of the labels. A {\em dominant
  majority rule} for binary prediction is one that outputs +1 or -1,
if one of the classes gets {\em significantly} more votes than any of
the others. If the number of votes for both labels are similar, the
dominant majority rule outputs ``?'' which is interpreted as ``I don't
know''.

\section{Augmenting medicine}

  \Medicine{Protocol limitation}{
  \sout{\yoav{For readers that are not MD, we should explain what are protocols, how they are generated, and whether all or some of their functionality ca be taken over by a computer. Also, I would put "extrapolation error" in here}}
  {\color{blue}According to NCI dictionaries, protocol means a detailed plan of a scientific or medical experiment, treatment, or procedure. \url{https://www.cancer.gov/publications/dictionaries/cancer-terms/def/protocol}. In clinics, it is a document that guides decision making, including criteria regarding diagnosis, management, and treatment. It exists in different areas of healthcare with different formats. In a loose sense, it could be understood as an algorithm solving a given mathematics problem. However, unlike the relationship between an algorithm and a mathematics problem, a medical protocol might not cover every situation and provide all possible solutions, and have several limitations.}
  
  
  The American Academy of Sleep
    Medicine (AASM) publishes criteria for manual sleep stage
    and
    sleep apnea annotation from the gold standard sleep study instrument, the polysomnogram (PSG). This annotation is based on manual analysis
    of biosignals recorded from the PSG \cite{Iber2007,berry2012aasm}. The AASM is a protocol that has been extensively applied, with rigorous scientific support, and updated regularly according to latest evidences. 
    %
    A detail sleep profile is critical for sleep quality enhancement, or even medical condition improvement.
    %
    However, it is well known that even with the well established protocol, the inter-rater agreement rate of sleep stage annotation among experienced experts, {\color{blue}in terms of percentage of epoch-by-epoch agreement, is only about 76\% over normal subjects and about 71\% over subjects with sleep apnea, while the Cohen's kappa is 65\% over normal subjects and about 59\% over subjects with sleep apnea}  \cite{norman2000interobserver}. Among many reasons, the one that is directly related to the intelligent system development is how the criteria are ``described'' in the protocol. For example, it is described in the protocol that if the delta wave occupies more than 20\% of a given 30-second epoch of the electroencephalogram during sleep, that 30-second epoch is defined to be the N3 stage. 20\% of a given 30-second epoch is 6 seconds. What about if the delta wave occupies 5.99-, or 6.01-seconds? What about if the delta wave sustains for 10 seconds, but it is divided into two consecutive 30-second epochs? When sitting on the ``gray area'' that is inherited from 
  the protocol, sleep experts need to make a
  decision based on their experience or the information
  they have at hand, and this leads to medical uncertainties, and hence the inter-rater, or even intra-rater disagreement.  
  
  {\color{blue}Another protocol limitation is the}
  ``extrapolation error''; that is, when we apply the developed
  protocol to the population different from the population that we
  collect the evidence for the protocol \cite{brosnan2015modest}. {\color{blue}Such extrapolation error usually comes from the variability among subjects. If such variability is big, it
  limits the development of a more quantitative protocol
  \cite{venhola2003interobserver}, and different protocols might be
  needed for different situations.}
}

The goal of this paper is to chart a path by which machine learning
and big data can be effectively used in medicine. Our discussion
focuses on the differences between AI and IA and we argue that IA
defines more achievable goals.

We use Ronald Heifetz classification of problems (see inset). Our
discussion up to this point focused on {\em technical}
problems. Technical issues can be solved by changing equipment or
directions for its use. In the remainder of this article we will
concern ourselves with the much harder {\em adaptive}
problems. Adaptive problems require changes in the perception and
behavior of {\em people}. In our case people are the caregivers who
are supposed to use the new technology. The best technology is
worthless if it is not adopted.

Consider first an AI system developed with the goal of replacing the
caregiver. Naturally, the caregiver will prefer not to use the
system. Will the patient prefer the human caregiver or the AI system?
The answer is anybody's guess, because such systems are not available
yet. Studies show that patients trust a human caregiver more than
technology \cite{ongena2020patients}, and that trust requires a human
to human connection \cite{nundy2019promoting}.

Unlike AI, IA systems do not aim to replace the caregiver. Rather,
they aim to assist the caregiver by taking care of the simple or easy
cases, and deferring to the human for the more complex cases. In this way IA can 
achieve the goal of ``computer and human work together'' \cite{verghese2018computer}. 
This makes the caregiver more efficient and effective, but does
not take away the agency and humanity of the caregiver. When the IA system outputs
``I don't know'' it explicitly passes the responsibility for the
patient to the caregiver.

%Agency and responsibility are central to patient-caregiver
%relationship.

As discussed earlier, medical cases often fall within gray areas, where
medics differ on the correct diagnosis or the best treatment. In such
cases the caregiver has the responsibility of making a decision even
though the decision might be wrong. Price et. al. have studied the ethical,
legal and regulatory aspects of using AI in
medicine.\cite{price2014black,ford2016privacy, price2017regulating}
Their conclusion is that the ultimate responsibility for the patients
well-being is {\em always} with the human caregiver. Even if the AI
system is known to make fewer mistakes than the average doctor,
mistakes are unavoidable, and the question is who is held responsible for
the mistake, and who needs to explain their decisions and potentially lose their license to practice.

\iffalse

In this book the point is made that human error is inevitable.
\yoav{ What is the main point of this paper? \cite{donaldson2000err}}

\sout{
Today, and in the foreseeable
future, \sout {\color{blue}non-trivial efforts are needed to convert} medicine \sout{will not be}{\color{blue}to} a precise science. {\color{blue}Even if medicine fulfills precise science,} incorrect decisions {\color{blue}is inevitable due to human natures \cite{donaldson2000err}, and }
can result in harm or death.  }
\fi

We give three perspectives on the possible integration of IA with
medicine: IA and the individual medic, the decentralization of
medicine, and IA as a method for disseminating medical knowledge.

\subsection{IA and the individual medic}

\Org{Technical Vs. Adaptive Problems}{ 
We quote from Robert Rechter's book "The digital doctor"~\cite{wachter2015digital}:
\begin{quote}
  Harvard psychiatrist and leadership guru Ronald Heifetz has
  described two types of problems: technical and adaptive. Technical
  problems can be solved with new tools, new practices, and
  conventional leadership. Baking a cake is a technical problem:
  follow the recipe and the results are likely to be fine. Heifetz
  contrasts technical problems with adaptive ones: problems that
  require people themselves to change. In adaptive problems, he
  explains, the people are both the problem and the
  solution. Leadership, he once said, requires mobilizing and engaging
  people around a problem “rather than trying to anesthetize them so
  you can go off and solve it on your own.”
\end{quote}

Rechter continues to say that the digitization of medicine "the Mother
of All Adaptive Problems". In other words, for AI to be widely
adapted, doctors and nurses ("medic" in the following) need to
positively engage in its adaptation. Declaring that AI will soon
replace medics, positions AI in an adversarial stance towards medics
and is likely to make them more resistant to the adoption of AI
technology~\cite{topol2019deep}.
} 
From the perspective of an individual medic, an IA sidekick is a
tool that augments their diagnostic abilities by increasing accuracy
and by saving time.

To better understand the diagnostic process and the possibilities  of
improving it using IA, we turn to the Kahaneman's~\cite{kahneman2011thinking}
``Thinking Fast Thinking Slow'' and to Vordermark book on medical
decision making~\cite{vordermark2019introduction}. According to these authorities,
medical diagnosis is a combination of two types of processes: {\em
  recognition} and {\em elimination}.

{\em Recognition} is an automatic mental process where one diagnosis
presents itself in the doctors mind as truth. Pathologists,
Radiologist's and other ``Pattern Doctors''~\cite{topol2019deep} make heavy use
of recognition. Their experience allows them to quickly sift through
large amounts of data and detect complex patterns. Pattern Doctors
often work under great time pressure, which can cause them to miss
important patterns. IA can help the doctor by performing a fast
analysis of the signal and alerting the doctor to locations that might
indicate a pathology. This improves the accuracy and speed of the
pathologist while maintaining the responsibility of the doctor to the
final diagnostics. 

Pattern doctors are often unable to fully explain their diagnostics.
This hinders documenting, critiquing and teaching diagnostics. As
recognition typically points to a single diagnosis, there is a danger
of overlooking other possible diagnoses. IA can serve as a ``note
taker'' documenting the diagnostic process, and pointing out possible
errors of omission. 

An example of a pattern classification problem is the annotation of
sleep. There is active research on using AI to automate this task
\cite{sleepHT2020}.
There is significant inter-rater agreement rate among experts for 
{\em normal subjects}, it is still far from perfect. Agreement drops significantly for abnormal subjects with subjects with sleep apnea   ... . 
Existing systems, however, output the identified sleep stages, 
without providing any measure of confidence. Providing a measure of confidence allows the HP to concentrate on the complex parts of the signal, while delegating the easy pars to IA sidekick. We believe that such division of labor will increase the likelihood that the HP would use the IA system.

An ideal IA
sidekick should let experts know "I don't know", where there is any
ambiguity in the process of annotation. Suppose there are 10\% "I
don't know" sleep stages, it could help save experts 90\% of time, and
focus on more important issues.

{\em Elimination}, unlike recognition, is a slow deliberative and
verbal process which starts with all possible diagnoses and gradually
eliminates unlikely ones based on patient history, examination and
test results. As Elimination is deliberative, it is easier to discuss,
document and teach it.

An IA system could help the elimination process carefully and
systematically eliminate diagnoses. This can help the doctor stay
aware of possibilities that are not obvious, for differential
diagnosis.


\subsection{IA and Decentralized medicine}

Medicine today is highly centralized. Most interactions
between patient and medic occur in hospitals and clinics. These
large facilities are expensive to build and to maintain. Traveling to
a hospital and back in order to see a doctor for 5 minutes is highly
inefficient. 

Part of the solution is telemedicine. In these days of Covid-19,
telemedicine has gained popularity. Medic and patient can meet in a
virtual space without either leaving home. Moreover, diagnostic
devices can be placed in the patient's home and provide the doctor with
vital signs.


Devices exist that allow most patients to measure basic vital signs
such as blood pressure, heart rate, EKG and temperature. More complex signals such as
polysomnogram or ultrasound currently require a local nurse or a
technician and/or a local facility outside the home.

IA can reduce the need for a nurse or technician in two ways. First,
it can guide the patients in placing the sensors on his/her
body so as to get a good signal or image. Second, it can perform an
initial diagnosis. If the patterns are simple, output a diagnosis.
Otherwise, output IDK and alert the remote healthcare provider. 

One significant disadvantage of decentralized medicine is the lack of monitoring. The patient might be non-complying in terms of taking medication, substance abuse or exercise without anybody knowing. IA agents can monitor the patient's activity and alert HP when non-compliance is detected.

Over-centralization is particularly problematic in long-term care.
Seniors are often pressured to move to institutions such as
assisted living so that they are closer to a medical staff. This, in
spite of significant medical, mental and financial cost of such a
move. An approach to long term care called ``aging in
place'' is gaining popularity around the world. IA sidekicks can help
seniors perform tasks without taking away their agency.


\subsection{IA and knowledge dissemination}

Through machine learning, IA sidekicks can adapt, over time, to the
medic using them. This is particularly true for pattern doctors
performing analysis of complex signals. Initially, the sidekick will
use some standard set of parameters which gives reasonable performance on
typical signals. Over time, the sidekick will learn to imitate the
doctor that is using it. At this point the sidekick captured some of
the recognition abilities of the doctor. The machine learning term for
the captured information is the ``learned model''.

Learned models, especially those corresponding to experienced and
successful doctors, are likely to be useful for other doctors,
possibly at the beginning of their career. Models from many doctors,
in many institutions, can be collected in repositories. Such models
will have many uses, from initializing sidekicks for new doctors,
through the integration of many models into a single better model.

IA models complement other methods of medical method dissemination
such as protocols, books, lectures and journal articles. Such models might have the advantage of capturing pattern recognition methods
which are often hard to describe in words. Training a cancer detection model is likely to require less effort and be more effective than writing a journal article that explains the same. \cite{reid2000medical}.

These models are likely to agree with each other on the easy cases, but are likely to disagree on the harder cases. By comparing the outputs of models trained by multiple experts on can distinguish the easy cases, on which a confident prediction can be made, from the harder cases, on which the diagnosis is IDK, and the doctor needs to use additional tests to arrive at a reliable diagnosis.

\section{About the Authors}
\begin{itemize}
\item {\bf Hau-Tieng Wu} is an associated professor in the Department
  of Mathematics and Department of Statistical Science at Duke
  University. Dr. Wu received his MD from National Yang Ming
  University (Taiwan) in 2003, and PhD in mathematics from Princeton
  University in 2011. The goal of Dr. Wu's research is to improve
  human health by developing rigorous and accurate algorithms
  quantifying human physiological dynamics.  Concretely, he focuses on
  the mathematical and statistical foundation of machine learning
  tools to transform multi-modal physiological wave-forms into
  diagnostically meaningful measures.
\item {\bf Yoav Freund} is a professor in the Department of Computer
  Science and Engineering in the university of California, San
  Diego. He received his PhD. in Computer Science in the University of
  California, Santa Cruz in 1993. He is best known for his Work with
  Robert Schapire on boosting, work for which they recieved the Godel
  prize in 2003 and the Kannelakis Prize in 2004.  His work is in
  machine learning theory and applications. His current focus is
  machine learning algorithms to large scale image data.
\end{itemize}

\section{References}
% \bibliographystyle{alpha} 
\bibliography{medbib}

\end{document}

