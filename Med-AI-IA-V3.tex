\documentclass[11pt]{pnas-new}
% \documentclass[10pt]{article}
\templatetype{pnasresearcharticle} % Choose template 
% {pnasresearcharticle} = Template for a two-column research article
% {pnasmathematics} %= Template for a one-column mathematics article
% {pnasinvited} %= Template for a PNAS invited submission

\usepackage[utf8]{inputenc}
\usepackage[T1]{fontenc}
%\usepackage{xcolor,ulem}
\usepackage{mdframed,ulem}
\usepackage{wrapfig}
\usepackage{multicol}
\setlength{\columnsep}{1cm}
\usepackage{graphicx}
\usepackage{adjustbox}
\usepackage{lipsum}
\newlength{\strutheight}
%\usepackage{soul} % for strike-through (\st)

\author[1]{Hau-Tieng Wu}
\author[2]{Yoav Freund}

\affil[1]{Duke, Mathematics and Statistical Science, Durham, 27708, USA}
\affil[2]{UCSD, Computer Science, San Diego, 92093, United States. yfreund@eng.ucsd.edu}


\title{Trusted digital doctors know their limits}

\newcommand{\comment}[3]{{\color{#1} {\bf #2 :} #3}}
%\newcommand{\comment}[3]{}  % suppress comments
\newcommand{\hautieng}[1]{\comment{blue}{hautieng}{#1}}
\newcommand{\yoav}[1]{\comment{red}{Yoav}{#1}}

\mdfsetup{middlelinecolor=blue, middlelinewidth=2pt, backgroundcolor=blue!10, roundcorner=10pt}

\mdfdefinestyle{Medicine}{middlelinecolor=red, middlelinewidth=2pt, backgroundcolor=red!10, roundcorner=10pt}
\mdfdefinestyle{ML}{middlelinecolor=green, middlelinewidth=2pt, backgroundcolor=green!10, roundcorner=10pt}
\mdfdefinestyle{Org}{middlelinecolor=blue, middlelinewidth=2pt, backgroundcolor=blue!10, roundcorner=10pt}

\newcommand{\block}[3]{\begin{mdframed}[style=#1]{\bf #2}\\ #3 \end{mdframed}}
\newcommand{\Medicine}[2]{\block{Medicine}{#1}{#2}}
\newcommand{\ML}[2]{\block{ML}{#1}{#2}}
\newcommand{\Org}[2]{\block{Org}{#1}{#2}}

%% \Medicine, \ML and \Org generate different colored boxes for short
%% explanations in different areas.
%% The command format is \ML{Title}{text}
%%
%% Example:
%%\ML{Title}{
%%  bla bla bla  bla bla bla bla bla bla bla bla bla
%%  bla bla bla  bla bla bla bla bla bla bla bla bla
%%}

\renewcommand{\input}[1]{}

\begin{abstract}

  The meteoric rise of AI in general and Deep Learning in particular
  is generating great excitement throughout academia and commerce, and
  in particular in medicine\cite{topol2019deep,
    wachter2015digital}. With some some high-profile claims
  that AI will soon replace humans in many medical specialties.

  In this position paper we present an alternative view. We contrast
  {\em Artificial Intelligence} with {\em Intelligence Augmentation}
  and argue that the second is more likely to benefit the patient than
  the first. We provide evidence to this argument and present a vision
  in which easier decisions are delegated to computers, while the more
  difficult ones are handled by humans.

\end{abstract}

\begin{document}
\settoheight{\strutheight}{\strut}

 
\maketitle

%\thispagestyle{firststyle}
\iffalse
\section{issues to be resolved}

\hautieng{As we discussed, I followed the medical journal convention and changed the author order so that you are the senior/corresponding author.}

\begin{itemize}
    \item Introduce the term "Healthcare Provider" with acronym HP, and use throughout instead of doctors nurses etc. \hautieng{I am not sure what is the best way to do it. In many places, "doctor" are clearly more precise than HP. So I leave this part to you.}
    \item Patient compliance - how can IA help make sure that patients take their meds, don’t drink excessively etc. \hautieng{done}
    \item Compensation: Doctors that generate quality data should own this data, distribution mechanisms should compensate the doctor for his/her contribution. \hautieng{As we discussed, this might be a bit off the topic. This is more like a regulatory issue. Let's discuss it if you have a different viewpoint.}
    \item A few grammatical corrections:
    -insert 2 page 2 last line “piopsies” instead of biopsies
    -last line page 2 spelling: “therefor” (needs an e) therefore 
    -insert page 7 we quote from Robert Rechter’s book... (53) Reference @ 53 indicates “Robert Wachter” as author. \hautieng{I am not fully sure what you want...}

\end{itemize}

To Highlight a change use this macro: \change{old text}{new text}
\fi

\section{Introduction}

The meteoric rise of  Artificial Intelligence (AI) and Deep learning raises the possibility that
doctors will be replaced by computers~\cite{Mukherjee2017}. Geoff Hinton,
a famous deep learning researcher said in 2017: ``It's just completely
obvious that in ten years deep learning is going to do better than
Radiologists ... They should stop training radiologists now''.

The predictions of Sebastian
Thrun~\cite{Mukherjee2017,esteva2017dermatologist}, another leader in
machine learning, are less disruptive: ``... deep learning devices
will not replace dermatologists and radiologists. They will {\em
  augment} professionals, offering the expertise and assistance''. 
  The term "Augmented Intelligence" appears frequently in recent 
  policy reports from medical associations~\cite{american2019augmented,AAD2019augmented}. Collaborations between computers and human care-givers are seen as a desirable goal.
  In this article we suggest how such a collaboration between human and computer might unfold in practice. Central to our approach is to allow the computer to quantify it's own uncertainty and, when appropriate, output "I don't know" (which we will abbreviate as IDK).
 
\Org{Artificial Intelligence and Intelligence Augmentation}{
The driving question of AI can be summarized as: ``are machines
  capable of behaving in a way that indistinguishable from that of
  humans, as judged by other humans''.  The archetypal test
  of whether artificial intelligence has been achieved is the {\em Turing
  Test}, in which a human, communicating with another agent through
  text alone, is unable to tell whether or not the agent is human. A
  natural consequence of computers being indistinguishable from humans
  is that they will be replacing humans, causing mass unemployment.

  The driving question of IA is whether and how computers can be used 
to {\em augment} humans rather than replace them. Some augmentations
are the territory of science fiction. For example,
cyborgs whose anatomy is part human, part artificial and can with
equal ease solve complex equations or write poetry. Other examples are
so mundane are so mundane that they are taken for granted. Examples
are the smart phone and google search
are ways in which our capabilities are augmented by computers. 
 
The Turing test was published~\cite{turing1951can} in 1951. A 1956
workshop in Dartmouth college is widely recognized as the beginning of
the field of AI. IA appeared on the scene soon thereafter, with
Ashby~\cite{ashby1957introduction} in 1957
Licklider~\cite{licklider1960man} in 1960 and
Englbart~\cite{engelbart1962augmenting} in 1962.

Arguably, the impact of IA on today's society is much larger than
that of AI. Siri, Google search and assisted driving are some of the
common apps that augment human ability. On the other hand, the goal of
creating a general purpose AI that possesses a human-like capability
to reason about new domains seems to be as far as ever.  At the same
time, AI holds the fascination of many, probably because of it's
tantalizing combination of promise and threat.}

 The argument as to whether human healthcare providers (HP)~\footnote{Our discussion applies to all healthcare providers, including Doctors, Nurse Practitioners and Nurses. To simplify the references to this group of professionals we use the acronym HP, which stands for "Healthcare Provider"}  will be replaced or empowered by computers is part of a wider debate between AI, whose goal is to imitate human behaviour  and Intelligence Amplification (IA), whose goal is to extend human capabilities. Computers with AI are sometimes called "AI agents", we contrast computers with IA by referring to them as "IA Helpers" (IAH). Helpers are expected to help when they can, and say IDK when they cannot. In any case, they are not allowed to take actions regarding the patient.
Of the many potential uses of AI/IA in medicine, we focus on diagnostics.
AI-driven diagnosis are expected to become a reality much sooner than other medical activities such as surgery~\cite{topol2019deep}. 

Central to our argument is a quantification of {\em prediction
  confidence}. Such quantification is needed to avoid premature
diagnostic conclusions, and to decide which additional tests or
consultations might be needed. Consider a doctor that is asked
to diagnose a patient with complex or conflicting symptoms. A careful
doctor will admit their uncertainty and perform additional tests or
ask a specialist. A less careful, overly self confident doctor is
more likely to give an incorrect diagnosis and choose an ineffective or even damaging
treatment plan.

An AI agent, expected to be better than the human HP, is more likely to 
end up behaving like an overly confident doctor. An IAH, aware of
its own limitations, will give advice only when the evidence is
strong and otherwise say IDK.

In the following we explore these ideas. We
start with a critique of one of the papers that claims that AI agents
can outperform human diagnosticians.

\section{Black-box learning}
\label{sec:ground-truth}

Deep learning (DNN) is a powerful black-box learning algorithm. Examples of applications of DNN to medical diagnosis include arrhythmia classification 
from single channel electrocardiogram \cite{hannun2019cardiologist}, diabetic retinopathy detection from retinal fundus photographs \cite{gulshan2016development}, skin cancer diagnostics~\cite{esteva2017dermatologist} as well as many others.

\ML{Supervised Learning and ground truth}{Roughly speaking, machine
  learning (ML) can be divided into {\em unsupervised} learning and
  {\em supervised} learning. In both cases, the task of the learning
  algorithm is transforming a set of {\em examples} into a {\em model}. In
  unsupervised learning, the examples are undifferentiated raw
  measurements. In supervised learning, which is the focus of
  this article, each example consists of an {\em input} and a {\em
    label}. Typically, the labels are provided by human
  experts.

\yoav{Ground truth is centeral to our comparison with AI and to our
criticism of the DNN papers}

These labels \sout{\color{red}are usually viewed
  as}\sout{define} correspond to the {\em ground truth} and the goal of
  the learning algorithm is to make predictions that diverge as little
  as possible from the ground truth.}


%The data for supervised learning consists of a large collection of
%(input, output) pairs. For medical diagnosis, usually the input is medical
%information for the patient (Heart rate, blood tests, X-ray images
%etc.) and the output is the diagnosis. This output is output is assumed to be the undisputed
%``ground-truth''. Unfortunately, this assumption is problematic in the context of medical diagnostics. 
%It is not uncommon that different diagnosticians give different diagnoses for the same medical %information which 
%makes it hard to assign a ground to each instance. We will return to this important issue in the next %section.

%The other important assumption made in supervised learning is that the
%generated classifier is tested using the same distribution of examples
%as that of the training set. In practice, however, this is often hard to achieve.

\ML{Skin cancer diagnosis using Deep Neural Networks}{One of the
  papers that provided evidence that deep neural networks might be
  able to outperform humans is the work of Esteva et
  al~\cite{esteva2017dermatologist}. They trained a Deep neural
  network to classify images of skin into three categories: benign,
  malignant and non-cancerous. The network was then tested, along with
  twenty five dermatologists on images which were labeled by a
  pathologist analysis of the biopsy. The neural network performed
  comparably to, and sometimes better than the human dermatologist.
  To provide ground truth, the patients were biopsied and the piopsies
  were diagnosed by pathologists.
}  

In a highly cited paper in the journal Science~\cite{esteva2017dermatologist} the authors 
makes the claim that DNNs can perform diagnostics as well as, or better, than board certified dermatologists. 

A fundamental problem with the experiments presented in the paper is that the test data used does not represent 
a realistic deployment of the system.
The data used in the experiment was {\em retrospective},
i.e., it was collected from the records of past patients for which both
a skin image and a biopsy were available. Normally, patients get
biopsied only if the dermatologist thinks there is a significant
chance of {\bf malignancy}. As a result, a retrospective study that is
based on patients for whom a biopsy was taken is likely to
over-represent malignant patients and therefore be biased. If an image-based classifier
is trained on the biased data, its performance on unbiased test data
is likely to be worse. Specifically, when the classifier is applied to skin
images of un-diagnosed patients it is likely to over-diagnose them as
malignant, potentially resulting in an increase of the number of unnecessary biopsies.

\section{Uncertainty in medicine}

In medical diagnostics the {\em ground truth} are often unavailable, and uncertainty in diagnosis is common. 

{\color{blue}One way to confirm a diagnosis, or obtain the ground truth, is to collect more information through additional examinations and tests. However, cases (patients) differ in their complexity, and the attending medic might judge additional tests to be unnecessary, in which case the outcome of these tests will not appear in the observational study. 
In {\em simple} cases, the initial exam is enough for the doctor to confidently choose a treatment. In a more complex case, the doctor might ask for multiple tests and visits, refer the patient to a specialist, consult colleagues, journals and books, etc, to narrow the set of likely diagnostics and choose a treatment plan. 
%
In addition, diagnosis is not a simple input-output mapping. Rather, it is an iterative process which reduces uncertainty over time. To illustrate this, consider the diagnostics of a patient that is treated in a clinic.  When a patient arrives at the clinic for the first time, all diagnostics are possible. After a physical exam and an interview with a doctor, many possibilities are eliminated.
%
Another way is following up on the treatment plan. However, correct diagnosis is only one of many factors that influence the long term health of the patient. Others include choice of treatment, medication adherence, changes in living and working environments, diet, stress, etc.





Given the difficulties in measuring an undisputed "ground truth" for diagnostics, we suggest that the goal of learning be set not to predicting a {single} "true" diagnostic, but rather to {taking} the actual diagnostics given by the HP as the ground truth. In other words, we change the goal from performing better than the human to that of performing almost as well as the humans. Which means that in order for IA to aid the HP in the diagnostic process, uncertainty or disagreement among the HP should be reflected as uncertainty in the output of the IAH, and 
 it has to have a way to express both what is knows and what it {\em does not} know.
%
This conforms to the goals of AI vs IA, while AI's goal is to replace the HP, the goal of IA is to aid the HP and make him/her more efficient and more accurate.



There are many causes for uncertainty in human medical diagnosis. One approach for quantifying uncertainty in medical diagnosis is to use inter-rater studies, {where} multiple doctors produce diagnostics based on identical medical information without communicating with each other. Associating ground truth with these diagnostics whose inter-rater agreement is less than perfect is challenging.
We briefly describe three categories of problems: {\em signal quality}, the {\em knowledge gap} and the limitations of {\em diagnostic protocols}.}

By {\em Signal Quality} refers to the quality of the raw data
collected for medical diagnosis. Some diagnostic measures, such
as heart rate, blood pressure and temperature can be measured more reliably
and consistently than other
measures such as camera images, X-ray and ultrasound, 
some of which might produce vast and highly variable data. The quality of this data
depends on many factors, including the quality of the instruments, the
consistency of the human operator, the build of the patient, etc.

\Medicine{Alarm fatigue}{ Patient monitors are bedside medical devices
  that monitor patients that are at risk but currently stable, freeing
  the medical staff to attend to the patients whose status is
  critical.  Unfortunately, Patient monitors suffer from signal
  quality issues and tend to generate false alarms at a high
  rate. Over time, this can result in the staff not responding to
  alarms, potentially resulting in great damage to the patient. This
  phenomenon, called {\em alarm fatigue} (or alarm overload) is a
  major problem in hospital care \cite{cvach2012monitor,brief2019top}.
  
  \yoav{Two questions: (1) Which of these papers are about the severity of the problem and which are about research? (2) regarding research, has there been research aimed at making the monitors adaptive to the patient and or nurse?}
  
  Alarm fatigue is a well known issue medical providers encounter when working with patient monitors. It is frequently named as a threat to patient safety \cite{sendelbach2013alarm,ruskin2015alarm}, and a lot of research has been carried out toward this problem \cite{cvach2012monitor,paine2016systematic,bai2016sequence,hu2019algorithm}.
}

%Signal quality enhancement is already an important part of imaging
%devices such as X-ray and MRI. Methods such as compressed
%sensing~\cite{lustig2008compressed} are used to reconstruct 3d images
%from a large number of noisy scans.

One situation where signal quality and signal variability cannot be
neglected is Patient Monitors.  The purpose of these devices is to
continuously monitor patients vital signs and alert HP
if a dangerous situation is detected. Unfortunately, the false alarm
rate of these devices is often high, which leads to ``alarm fatigue'' where the medical staff ignores
the generated alarms, significantly reducing their utility.

Signal quality and alarm fatigue can be thought of as ``bottom up''
causes of uncertainty. The uncertainty originates in the medical
devices and moves up to the HP.

Other types of uncertainty are ``top down'' in that they originate in medical research percolates down to the HP. We briefly
describe two types of top-down uncertainty: knowledge gaps and the
limitation of medical protocols.

{\em ``Knowledge gap''} corresponds to limitations of scientific medical
knowledge. This is not the limitation of a particular doctor; rather,
it reflects the limitations of knowledge that correspond to successful
medical trials.
   \Medicine{knowledge gap}{ 
  
  
  
  {\color{blue} At the writing of this article the COVID-19 is a worldwide crisis \cite{sohrabi2020world}. Back in Jan 2020, when it was first reported in China, nobody had a clue how it will generate damage to human body, not to mention how to treat a patient. All medical practices, ranging from diagnosis to treatment to vaccine were all made based on experience. For example, the quinine and remdesivir were considered potential for hospitalized patients. A lot of data could be collected, but there is a huge gap between the data and what's going on. Knowledge was quickly accumulated in the past few months. For example, we know more about hydroxychloroquine, remdesivir, and many other candidate drugs; see, for example \cite{sanders2020pharmacologic,goldman2020remdesivir}. However, due to the lack of knowledge, even if there exist some protocols, for example \cite{world2020population,world2020protocol,nakajima2020covid,awad2020perioperative} and several others, there are still many white and unknown details in the overall medical practice.} 
  
  Knowledge gap is not confined to newly emerging diseases. Urodynamics can trace its root back to the 1800s \cite{perez1992history}, and recently become exponentially important in clinics. 
  %
  Urodynamic studies provide the best bladder and sphincter functional data for urologists to decide how to treat patients at risk for renal damage \cite{abrams2003describing}, or the diagnostic information for the cause and nature of a patient's incontinence, frequent urination, recurrent urinary tract infections, etc. 
  %
  While it has been extensively studied and applied in clinics, the main issue that plagues this field of urodynamics is the lack of precise definition of a detrusor contraction or overactive contraction \cite{abrams2003describing}.
  %
  For example, the protocols available today for reading the time series corrected from urodynamics, like International Children's Continence Society and International Continence Society, are mixed up by descriptive and quantitative statements \cite{bauer2015international,drake2018fundamentals}. 
%  
  The lack of a well defined definition is due to the lack of quantitative study from the pathophysiological perspective. For example, usually an overactive contraction represents itself as a ``bump'' in the detrusor pressure signal. However, what is the breath and height, or the shape, of a bump should we call it an overactive contraction? How to distinguish a true overactive contraction from an artifact? While there have been several reference information, like abdominal pressure, that could help us identify artifacts, but it can only explain a small portion of them. 
%
  Unsurprisingly, this fundamental issue has led to a significant
  inter-rater disagreement
  \cite{venhola2003interobserver,dudley2018interrater}.
  }



Even when medical knowledge exists, an individual doctor might not
have it. The dissemination of up to date and reliable medical
information is uneven. One of the most important information
dissemination tools are {\em medical protocols}. Those are used to
ensure uniformity and consistency of treatment between hospitals,
doctors and nurses. While protocols are an important dissemination
tool, they might not be available for all conditions.

\section{Quantifying uncertainty}

Medical cases vary in their complexity. When the human experts do not agree, it is unlikely that the IA helper will be able to resolve the disagreement. On the other hand, IA is likely to be more helpful on simple case, where doctors are likely to agree. However, the IA needs to {\em know} whether the case is easy or hard, so that it gives advice only when useful. How can the IA quantify it's own confidence?

Increasing confidence in a diagnosis by seeking consensus among
several doctors is common sense. A similar approach has been used in
machine learning algorithms such as 
Bagging~\cite{breiman1996bagging}, Random Forests~\cite{breiman2001random}, and
Boosting\cite{SchapireFr2012}. These so-called {\em ensemble} algorithms
take the majority vote of predictions from several ``base'' learning
algorithms using a majority vote to generate a single more reliable
prediction.
%~\footnote{A majority is used when there are only two possible labels. A more general combination rule is the {\em plurality} i.e., the label that gets the largest number of votes.}

%The simple majority vote always outputs one of the labels. A standard technique for measuring the %confidence of the prediction is to consider the difference in number of votes between the two top %classes (diagnostics). 
%If the difference is large, then the top class is output. If it is small, then the algorithm outputs %IDK.


\section{Augmenting medicine}

  \Medicine{Protocol limitation}{

Protocols provide a rigorous method for disseminating best practices
in medicine. In essence a medical protocol is a detailed procedure,
recipe, or algorithm for treating a particular disease.

Protocols are usually formulated by committees of experienced
  healthcare providers.  The dissemination of protocols
  unifies and standardizes the medical workflow. This standardization
  reduces errors, enhances reproducibility and
  defines a standard of care.
  
However, protocols are made to be read an interpreted by caregivers,
as such it is extremely hard to make them precise. 
Protocols, being written in a common language, such as
English, french or mandarin, are inherently imprecise and can be understood
differently by different doctors, This can lead to inconsistent diagnosis.

For example, consider sleep analysis. A detail sleep profile is
critical for sleep quality enhancement, or even medical condition
improvement.
The American Academy of Sleep
  Medicine (AASM) publishes criteria for manual sleep stage and sleep
  apnea annotation from the gold standard sleep study instrument, the
  polysomnogram (PSG). This annotation is based on manual analysis of
  biosignals recorded from the PSG \cite{Iber2007,berry2012aasm}. The
  AASM is a protocol that has been extensively applied, with rigorous
  scientific support, and updated regularly according to the latest studies.

However, it is well known that even with a well established
  protocol, the inter-rater agreement rate of sleep stage annotation
  among experts, in terms of percentage of
  epoch-by-epoch agreement, is only about 76\% over normal subjects
  and about 71\% over subjects with sleep apnea, while the Cohen's
  kappa is 65\% over normal subjects and about 59\% over subjects with
  sleep apnea \cite{norman2000interobserver}. Among many reasons, the
  one that is directly related to the intelligent system development
  is how the criteria are defined in the protocol.

For example,
  it is described in the protocol that if the delta wave occupies more
  than 20\% of a given 30-second epoch of the electroencephalogram
  during sleep, that 30-second epoch is defined to be the N3 stage. However, the determination of the duration of delta wave is subjective.
  It is up to the sleep
  expert to make a
  decision based on their experience or other information they have at
  hand. 
  
  Another protocol limitation is
  ``extrapolation error'' which  occurs when a protocol that was
  developed based on studies in one population is applied to a very
  different population~\cite{brosnan2015modest}. Developing protocols
  that can be applied world-wide requires careful experimental design
  to ensure that samples are representative of world
  population.~\cite{venhola2003interobserver}.
}

Our focus so far was on 
the {\em technology}, not on the {\em adoption} of the technology by humans and organizations. While technology can advance very quickly, the adoption of technology is can be slow and difficult.  
We devote this section to discussing the adoption of AI and IA in medicine.

Consider first an AI system developed with the goal of replacing the
caregiver. Naturally, HP will prefer not to use the
system. Will the patient prefer the human HP or the AI system?
Some preliminary studies show that patients trust a human HP more than
technology \cite{ongena2020patients}, and that trust requires a human
to human connection \cite{nundy2019promoting}.

Unlike AI, IA systems do not aim to replace HP. Rather,
they aim to assist HP by automating the simple
cases, and deferring to the human for the more complex cases. In this way IA can 
achieve the goal of ``computer and human work together'' \cite{verghese2018computer}. 
This makes HP more efficient and effective, but does
not take away the HP's agency or humanity. When the IA system outputs
IDK it explicitly passes the responsibility for the
patient to HP.

As discussed earlier, challenging medical cases often fall within gray areas, where
HPs differ on the correct diagnosis or the best treatment. In such
cases HP has the responsibility of making a decision even
though the decision might be wrong. Price et. al. have studied the ethical,
legal and regulatory aspects of using AI in
medicine.\cite{price2014black,ford2016privacy, price2017regulating}
Their conclusion is that the ultimate responsibility for the patients
well-being is {\em always} with the human HP. Even if the AI
system is known to make fewer mistakes than the average doctor in some well-defined tasks,
mistakes are unavoidable, and the question is who is responsible for
the mistake, and who needs to explain their decisions and potentially lose their license to practice.

\iffalse

In this book the point is made that human error is inevitable.
\yoav{ What is the main point of this paper? \cite{donaldson2000err}}

\sout{
Today, and in the foreseeable
future, \sout {\color{blue}non-trivial efforts are needed to convert} medicine \sout{will not be}{\color{blue}to} a precise science. {\color{blue}Even if medicine fulfills precise science,} incorrect decisions {\color{blue}is inevitable due to human natures \cite{donaldson2000err}, and }
can result in harm or death.  }
\fi

We give three perspectives on the possible integration of IA with
medicine: IA and the individual medic, the decentralization of
medicine, and IA as a method for disseminating medical knowledge.

\subsection{IA and the individual HP}

\Org{Technical Vs. Adaptive Problems}{ 
We quote from Robert Wachter's book "The digital doctor"~\cite{wachter2015digital}:
\begin{quote}
  Harvard psychiatrist and leadership guru Ronald Heifetz has
  described two types of problems: technical and adaptive. Technical
  problems can be solved with new tools, new practices, and
  conventional leadership. Baking a cake is a technical problem:
  follow the recipe and the results are likely to be fine. Heifetz
  contrasts technical problems with adaptive ones: problems that
  require people themselves to change. In adaptive problems, he
  explains, the people are both the problem and the
  solution. Leadership, he once said, requires mobilizing and engaging
  people around a problem “rather than trying to anesthetize them so
  you can go off and solve it on your own.”
\end{quote}

Wachter continues to say that the digitization of medicine is "the Mother
of All Adaptive Problems". In other words, for AI to be widely
adapted, doctors, nurses and other HP  need to
positively engage in its adaptation. Declaring that AI will soon
replace medics, places AI in a counter-productive adversarial stance towards HP
and is likely to delay its adoption~\cite{topol2019deep}.
} 
From the perspective of an individual medic, an IA helper is a
tool that augments their diagnostic abilities by increasing accuracy
and by saving time.

To better understand the diagnostic process and the possibilities  of
improving it using IA, we turn to the Kahaneman's~\cite{kahneman2011thinking}
``Thinking Fast Thinking Slow'' and to Vordermark book on medical
decision making~\cite{vordermark2019introduction}. According to these authorities,
medical diagnosis is a combination of two types of processes: {\em
  recognition} and {\em elimination}.

{\em Recognition} is an automatic mental process where one diagnosis
presents itself in the doctors mind as truth. Pathologists,
Radiologist's and other ``Pattern Doctors''~\cite{topol2019deep} make heavy use
of recognition \yoav{\sout{when the task is about recognizing patterns}}. Their experience allows them to quickly sift through
large amounts of data and detect complex patterns. Pattern Doctors \yoav{\sout{handling patterns}}
often work under great time pressure, which can cause them to miss
important patterns. IA can help the doctor by performing a fast
analysis of the signal and alerting the doctor to locations that might
indicate a pathology. This improves the accuracy and speed of the
pathologist while maintaining the responsibility of the doctor to the
final diagnostics. 

As recognition is often a automatic mental process it can be difficult to explain verbally.
This hinders documenting, critiquing and teaching pattern recognition. As
recognition typically points to a single diagnosis, there is a danger
of overlooking other possible diagnoses. IA can serve as a ``note
taker'' documenting the diagnostic process, and pointing out possible
errors of omission. 

An example of a pattern classification problem is the annotation of
sleep. There is active research on using AI to automate this task
\cite{sleepHT2020}.
There is non-trivial inter-rater disagreement rate among experts.  
Existing systems, however, output the identified sleep stages, 
without providing any measure of confidence. Providing a measure of confidence allows the HP to concentrate on the complex parts of the signal, while delegating the easy pars to IAHs. 
Suppose 10\% of the cases result in IDK. HPs could spend only a short amount of time on the remaining 90\% of the cases, and focus on more complex ones.
Such division of labor can increase the likelihood that a pattern HP will  the IA system.

{\em Elimination}, unlike recognition, is a slow deliberative and
verbal process which starts with all possible diagnoses and gradually
eliminates unlikely ones based on patient history, examination and
test results. As Elimination is deliberative, it is easier to discuss,
document and teach it.

An IA system could help the elimination process carefully and
systematically eliminate incorrect diagnoses. This can reduce the chance of overlooking possible diagnostics.

\subsection{IA and Decentralized medicine}

Medicine today is highly centralized. Most interactions
between patient and medic occur in hospitals and clinics. These
large facilities are expensive to build and to maintain. Traveling to
a hospital and back in order to see a doctor for 5 minutes is highly
inefficient. 

Part of the solution is telemedicine. In these days of Covid-19,
telemedicine has gained popularity. Medic and patient can meet in a
virtual space without either leaving home. Moreover, diagnostic
devices can be placed in the patient's home and provide the doctor with
{a record of vital signs during and between meetings.}


IA can surrogate a nurse or technician in telemedicine in two ways. First,
it can guide the patients in placing the sensors on his/her
body so as to get a good signal or image. Second, it can perform an
initial diagnosis. If the patterns are simple, output a diagnosis.
Otherwise, output IDK and alert the remote healthcare provider. 

One significant disadvantage of decentralized medicine is {limited} monitoring, especially when the patient is living alone. 
Falls and other accidents can go undetected for hours, days, or weeks, endangering the patient. IAHs can monitor the patient's activity, detect critical events and alert HP when needed.

Over-centralization is particularly problematic in long-term care.
Seniors are often pressured to move to institutions such as
assisted living so that they are closer to a medical staff. This, in
spite of significant medical, mental and financial cost of such a
move. An approach to long term care called ``aging in
place'' is gaining popularity around the world. IA helpers can help
seniors perform tasks without taking away their agency.


\subsection{IA and knowledge dissemination}

Through machine learning, IA helpers can adapt, over time, to the
medic using them. This is particularly true for  doctors that
perform pattern analysis of complex signals. Initially, the helper will
use some standard set of parameters which gives reasonable performance on
typical signals. Over time, the helper will learn to imitate the
doctor that is using it. This knowledge is captured in the  {\em learned model}.

Learned models, especially those corresponding to experienced and
successful doctors, are likely to be useful for other doctors,
possibly at the beginning of their career. Models from many doctors,
in many institutions, can be collected in repositories. Such models
will have many uses, from initializing helpers for new doctors,
through the integration of many models into a single better model.

IA models complement other methods of medical method dissemination
such as protocols, books, lectures and journal articles. Such models might have the advantage of capturing pattern recognition methods
which are often hard to describe in words. Training a pathologist by interacting with a cancer detection model is likely to be more effective than following a tutorial that explains the same \cite{reid2000medical}.

These models are likely to agree with each other on the easy cases, but are likely to disagree on the harder cases. By comparing the outputs of models trained by multiple experts one can distinguish the easy cases, on which a confident prediction can be made, from the harder cases, on which the diagnosis is IDK, and the doctor needs to use additional tests to arrive at a reliable diagnosis.

\section{About the Authors}
\begin{itemize}
\item {\bf Yoav Freund} is a professor in the Department of Computer
  Science and Engineering in the university of California, San
  Diego. He received his PhD. in Computer Science in the University of
  California, Santa Cruz in 1993.  He worked in Bell Labs, AT\&T Labs
  and Columbia University before joining UCSD in 2006. His work is in
  machine learning theory and applications. He is best known for his
  joint work with Robert Schapire on boosting, work for which they recieved
  the Godel prize in 2003 and the Kannelakis Prize in 2004.

\item {\bf Hau-Tieng Wu} is an associated professor in the Department
  of Mathematics and Department of Statistical Science at Duke
  University. Dr. Wu received his MD from National Yang Ming
  University (Taiwan) in 2003, and PhD in mathematics from Princeton
  University in 2011. The goal of Dr. Wu's research is to improve
  human health by developing rigorous and accurate algorithms
  quantifying human physiological dynamics.  Concretely, he focuses on
  the mathematical and statistical foundation of machine learning
  tools to transform multi-modal physiological wave-forms into
  diagnostically meaningful measures.
\end{itemize}

\section{References}
% \bibliographystyle{alpha} 
\bibliography{medbib}

\end{document}

