  \Medicine{Protocol limitation}{

Protocols provide a rigorous method for disseminating best practices
in medicine. In essense a medical protocol is a detailed procedure,
recipe, or algorithm for treating a particular disease.

Protocols are usually formulated by committees of experienced
  healthcare providers.  The dissemination of protocols
  unifies and standardizes the medical workflow. This standardization
  reduces confusion and omission, enhances reproducibility and
  provides a standard of care.
  
However, protocols, being written in human language, can be understood
differently by different doctors, This can lead to inconsistent diagnosis

For example, consider sleep analysis. A detail sleep profile is
critical for sleep quality enhancement, or even medical condition
improvement \yoav{Sleep Apnea? The sleep diagnosis paragraphs should
be ade tighter.}

The American Academy of Sleep
  Medicine (AASM) publishes criteria for manual sleep stage and sleep
  apnea annotation from the gold standard sleep study instrument, the
  polysomnogram (PSG). This annotation is based on manual analysis of
  biosignals recorded from the PSG \cite{Iber2007,berry2012aasm}. The
  AASM is a protocol that has been extensively applied, with rigorous
  scientific support, and updated regularly according to the latest studies.

However, it is well known that even with a well established
  protocol, the inter-rater agreement rate of sleep stage annotation
  among experienced experts, in terms of percentage of
  epoch-by-epoch agreement, is only about 76\% over normal subjects
  and about 71\% over subjects with sleep apnea, while the Cohen's
  kappa is 65\% over normal subjects and about 59\% over subjects with
  sleep apnea \cite{norman2000interobserver}. Among many reasons, the
  one that is directly related to the intelligent system development
  is how the criteria are defined in the protocol.

For example,
  it is described in the protocol that if the delta wave occupies more
  than 20\% of a given 30-second epoch of the electroencephalogram
  during sleep, that 30-second epoch is defined to be the N3
  stage. 20\% of a given 30-second epoch is 6 seconds. What about if
  the delta wave occupies 5.99-, or 6.01-seconds? What about if the
  delta wave sustains for 10 seconds, but it is divided into two
  consecutive 30-second epochs? The protocol definition is unreliable
  for such borderline or ``gray area'' cases. It is up to the sleep
  expert to make a
  decision based on their experience or other information they have at
  hand. This variability leads to medical uncertainties, and to high
  levels of inter-rater, or even intra-rater disagreement.
  
  Another protocol limitation is
  ``extrapolation error'' whcih  occurs when a protocol that was
  developed based on studies in one population is applied to a very
  different population~\cite{brosnan2015modest}. Developing protocols
  that can be applied world-wide requires careful experimental design
  to ensure that samples are representative of world
  population.~\cite{venhola2003interobserver}.
}
