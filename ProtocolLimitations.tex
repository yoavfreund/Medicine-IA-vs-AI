  \Medicine{Protocol limitation}{
  \sout{\yoav{For readers that are not MD, we should explain what are protocols, how they are generated, and whether all or some of their functionality ca be taken over by a computer. Also, I would put "extrapolation error" in here}}
  {\color{blue}According to NCI dictionaries, protocol means a detailed plan of a scientific or medical experiment, treatment, or procedure. \url{https://www.cancer.gov/publications/dictionaries/cancer-terms/def/protocol}. In clinics, it is a document that guides decision making, including criteria regarding diagnosis, management, and treatment. It exists in different areas of healthcare with different formats. In a loose sense, it could be understood as an algorithm solving a given mathematics problem. However, unlike the relationship between an algorithm and a mathematics problem, a medical protocol might not cover every situation and provide all possible solutions, and have several limitations.}
  
  
  The American Academy of Sleep
    Medicine (AASM) publishes criteria for manual sleep stage
    and
    sleep apnea annotation from the gold standard sleep study instrument, the polysomnogram (PSG). This annotation is based on manual analysis
    of biosignals recorded from the PSG \cite{Iber2007,berry2012aasm}. The AASM is a protocol that has been extensively applied, with rigorous scientific support, and updated regularly according to latest evidences. 
    %
    A detail sleep profile is critical for sleep quality enhancement, or even medical condition improvement.
    %
    However, it is well known that even with the well established protocol, the inter-rater agreement rate of sleep stage annotation among experienced experts, {\color{blue}in terms of percentage of epoch-by-epoch agreement, is only about 76\% over normal subjects and about 71\% over subjects with sleep apnea, while the Cohen's kappa is 65\% over normal subjects and about 59\% over subjects with sleep apnea}  \cite{norman2000interobserver}. Among many reasons, the one that is directly related to the intelligent system development is how the criteria are ``described'' in the protocol. For example, it is described in the protocol that if the delta wave occupies more than 20\% of a given 30-second epoch of the electroencephalogram during sleep, that 30-second epoch is defined to be the N3 stage. 20\% of a given 30-second epoch is 6 seconds. What about if the delta wave occupies 5.99-, or 6.01-seconds? What about if the delta wave sustains for 10 seconds, but it is divided into two consecutive 30-second epochs? When sitting on the ``gray area'' that is inherited from 
  the protocol, sleep experts need to make a
  decision based on their experience or the information
  they have at hand, and this leads to medical uncertainties, and hence the inter-rater, or even intra-rater disagreement.  
  
  {\color{blue}Another protocol limitation is the}
  ``extrapolation error''; that is, when we apply the developed
  protocol to the population different from the population that we
  collect the evidence for the protocol \cite{brosnan2015modest}. {\color{blue}Such extrapolation error usually comes from the variability among subjects. If such variability is big, it
  limits the development of a more quantitative protocol
  \cite{venhola2003interobserver}, and different protocols might be
  needed for different situations.}
}
