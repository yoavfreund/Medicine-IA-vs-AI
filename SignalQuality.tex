\Medicine{Uncertainty due to signal quality}{Medical devices use a
  variety of bio-sensors that record, display and distribute different
  biometric signals, ranging from vital signs such as heart rate,
  oxygen saturation, temperature and blood pressure, to high-frequency
  waveforms such as ECG, EEG, respiratory signal and arterial blood
  pressure. High frequency signals suffer from artifacts and other
  signal quality problems, some of which depend on the patient. In
  some cases, these problems can be handled by the human
  diagnostician. In other cases such as {\color{red}artifact removal of EEG is
    still an active research problem~\cite{islam2016methods}}.

  Another common source of signal quality issue is how
    the sensor is placed. While there have been several standards,
    ranging from the well-known ECG systems
    \cite{goldberger2017clinical} and {\color{blue}international 10–20
      EEG systems} \cite{Klem1999TheTE} to recently smart clothing
    system for telemedicine \cite{nesenbergs2016architecture}, it is
    not always possible to achieve a precise sensor placement for
    biomedical signal collection due to various reasons. This
    uncertainty might be tolerable for some clinical applications; for
    example, an imprecise ECG sensor placement might not impact the
    identification of some types of arrhythmia from the ECG signal,
    like atrial fibrillation. However, this uncertainty might cause
    troubles in other applications; {\color{blue}for example, an
      imprecise placement of the deep brain stimulation lead inside
      subthalamic nucleus might downgrade the Parkinson disease
      treatment outcome.}}
