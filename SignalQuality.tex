\Medicine{Uncertainty due to signal quality}{Medical devices use a
variety of bio-sensors that measure record and analyze different
  biometric signals. These signals vary in velocity, {\color{red}ranging} from low-velocity vital signs such as heart rate,
  oxygen saturation, temperature and blood pressure, through
  high-velocity waveforms such as ECG and EEG {\color{red}and medical} imaging such as CT, X-ray, MRI and
  scanning microscopes. {\color{red}Some} signals, such as blood
  pressure or heart rate, can be directly used in diagnosis. \sout{Mid- and 
  high-velocity} {\color{red}Some other} signals have to be interpreted before they can be used
  in diagnosis. We use the acronym HS to refer to \sout{mid and high
  velocity} those signals that require interpretation.

Interpreting HS is a significant fraction of the work of most
doctors. In addition, There are medical \sout{specializations}{\color{red}specialties} such as
Radiology and Pathology that are devoted to interpreting HS. These
so-called ``Pattern Doctors'' are predicted to be the early adopters
of AI~\cite{topol2019deep} or IA.
HS provides critical detailed information about the patient's
health. However, the richness of the signal can make it susceptible to
nuisance variability from noise, limited resolution, operator error,
etc. The number of nuisance variables is very large {\color{red}and versatile}. In the next
paragraph we give an example of one nuisance variable: the placement
of ECG leads on the patient's body.
  
For ECG signals to be correctly and consistently interpreted, it is
important that the leads are placed correctly on the patient's
body. Several standards for placement have been published, for example, the standard 12 leads ECG system
\cite{goldberger2017clinical}, the EASI system \cite{Dower1988}, and the Frank lead system \cite{frank1956accurate}. 
It is not always possible to achieve a consistent and precise sensor
placement for biomedical signal collection due to various reasons, for example, the torso variation caused by gender, age and living styles.
This uncertainty might be tolerable for some clinical applications;
for example, an imprecise ECG sensor placement might not impact the
identification of some types of arrhythmia from the ECG signal, like
atrial fibrillation. However, this uncertainty might cause troubles in
identifying other types of arrhythmia, for example, premature atrial contraction.

}
