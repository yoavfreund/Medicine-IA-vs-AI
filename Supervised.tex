\ML{Supervised Learning and ground truth}{Roughly speaking, machine
  learning (ML) can be divided into {\em unsupervised} learning and
  {\em supervised} learning. Deep learning method are mostly (but not exclusively) a special case of supervised learning.
  
  The task of the learning
  algorithm, in both supervised and unsupervised learning, is transforming a set of 
  {\em examples} into a {\em model} \yoav{which can be used to predict
  some aspect of new examples}. In
  unsupervised learning, the examples are unlabeled raw
  measurements. In supervised learning each example consists of an {\em input} and a {\em
    label}. Typically, the labels are provided by human
  experts. \yoav{These labels define the {\em ground truth}. The goal of
  learning is to generate a model that predicts the ground truth
  labels. The ground truth is used twice: to learn the model and to
  test the model. The availability of large and unbiased sets of
  examples with their ground-truth labels is critical for successful
  supervised learning and lack thereof can render supervised learning impossible

  Typical labels in the context of machine learning for
  medicine are Cancer/no-Cancer or diabetes/no-diabetes. In section 2
  we explain why ground truth labels are often not available,
  rendinging supervised learning impractical.}
}

