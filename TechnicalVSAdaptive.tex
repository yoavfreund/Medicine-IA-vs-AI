\Org{Technical Vs. Adaptive Problems}{ 
We quote from Robert Rechter's book "The digital doctor"~\cite{wachter2015digital}:
\begin{quote}
  Harvard psychiatrist and leadership guru Ronald Heifetz has
  described two types of problems: technical and adaptive. Technical
  problems can be solved with new tools, new practices, and
  conventional leadership. Baking a cake is a technical problem:
  follow the recipe and the results are likely to be fine. Heifetz
  contrasts technical problems with adaptive ones: problems that
  require people themselves to change. In adaptive problems, he
  explains, the people are both the problem and the
  solution. Leadership, he once said, requires mobilizing and engaging
  people around a problem “rather than trying to anesthetize them so
  you can go off and solve it on your own.”
\end{quote}

Rechter continues to say that the digitization of medicine is "the Mother
of All Adaptive Problems". In other words, for AI to be widely
adapted, doctors, nurses {\color{blue}and other caregivers} ("medic" in the following) need to
positively engage in its adaptation. Declaring that AI will soon
replace medics, positions AI in an adversarial stance towards medics
and is likely to make them more resistant to the adoption of AI
technology~\cite{topol2019deep}.
}