\ML{Arrhythmia detection using Deep
  Neural Networks}{Another paper supporting the clam that that deep
  neural networks can outperform humans is the work of
  Hannun et al~\cite{hannun2019cardiologist}. They trained a deep
  neural network to classify single-lead electrocardiogram (ECG) into 12 rhythm
  classes, including ten arrhythmias, normal sinus rhythm and noise.

Classifying single-lead ECG recordings is a challanging task even for
experienced cardiologists inter-rater agreement is low. Unlike~\cite{}
there is no simple way to objectively measure ground truth. Instead,
Hannun et al relied on a {\em concensus committee} of experienced
cardiologists, to define ground truth. A concensus committee
deliberates each recording until they reach a concensus
classification. Note that this type of concensus is very different
from perfect agreement in an inter-rater agreement study in which each
diagnostician has to commit to a diagnosis independently, without
communicating with others.

\yoav{this seems neither here nor there}
\st{In this study, the authors prepared a homemade database consisting
of single-channel ECG, and this database is much larger than the
publicly available database.}  

\yoav{How many cardiologists in the committee?}

With ground truth labels determined by a cardiologist consensus
  meeting, the authors showed that the DNN outperforms 6
  board-certificated cardiologists {\em outside the committee}.
  While this result is impressive, it is not clear whether the success
  can be attributable to the DNN. A different interpretation is that
  concensus committees make classifications that are significantly different from
  those of individual cardiologists. While it stands to reason that
  concensus committees are {\em better} than individual cardiologists,
  the lack of ground truth makes the last claim hard to verify.
 }
