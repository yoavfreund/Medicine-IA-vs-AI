\documentclass[10pt]{article}
\usepackage[margin=0.5in]{geometry}
\usepackage{float}
\usepackage{lipsum} % for dummy text only
% \usepackage[alpine,misc]{ifsym}
\textheight=9.0in
\pagestyle{empty}
\setlength{\tabcolsep}{0in}
\usepackage{hyperref}

    \newcommand\invisiblesection[1]{%
    \refstepcounter{section}%
    \addcontentsline{toc}{section}{\protect\numberline{\thesection}#1}%
    \sectionmark{#1}}

    \input{insbox}
    \usepackage[x11names]{xcolor}

    \begin{document}

    {\Huge Title}\\
    \textit{\small subtitle}

    \pagebreak[1]
    \section*{Section 1}

    \InsertBoxR{0}{
    \footnotesize\setlength\fboxsep{10pt}\setlength\fboxrule{1pt}
    \fcolorbox{IndianRed3}{AntiqueWhite1!30}{\begin{minipage}{2.1in}
    \invisiblesection{\textit{Side Bar}}
    \subsubsection*{AI vs IA}
    %
      {Using computers to augment human intelligence rather replace it is
      both tantalizing and mundane. On the heady side, consider
      cyborgs whose anatomy is part human, part artificial and can with
      equal ease solve complex equations or write poetry. On the mundane
      side, think of smartphones that are quickly becoming an inseparable
      part of our person.
       The idea of using computers to augment or amplify human intelligence
      has a very long history. The acronyms AI (Artificial intelligence) and
      IA (Intelligence Amplification or Intelligence Augmentation) have both
      become popular in the early
      1960's\cite{ashby1957introduction,engelbart1962augmenting}. These
      days, the acronym AI is popular, while the acronym IA is not. However,
      Sebastian Thrun's statement indicates that the idea of Intelligence
      augmentation is still on people's mind.}
    \end{minipage}}
    }[10]
    \lipsum[1-4]

    \pagebreak[1]
    \section*{Section 2}

    \lipsum[1-4]
    \end{document} 