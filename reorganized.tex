\documentclass[fleqn,10pt]{wlscirep}
\usepackage[utf8]{inputenc}
\usepackage[T1]{fontenc}
\title{Why the digital Doctor needs to say "I don't know"}

\author[1]{Yoav Freund}
\author[2]{Hau-Tieng Wu}
\affil[1]{UCSD, department, city, postcode, country}
\affil[2]{Duke, department, city, postcode, country}

%\keywords{Keyword1, Keyword2, Keyword3}



\begin{abstract}

  The meteoric rise of AI in general and Deep Learning in particular
  is generating great excitement throughout academia and commerce, and
  in particular in medicine\cite{topol2019deep,
    wachter2015digital}. With some some high-profile claims~\cite{}
  that AI will soon replace humans in many medical specialties.

  In this position paper we present an alternative view. We contrast
  {\em Artificial Intelligence} with {\em Intelligence Augmentation}
  and argue that the second is more likely to benefit the patient than
  the first. We provide evidence to this argument and present a vision
  in which easier decisions are delegated to computers, while the more
  difficult ones are handled by humans.

\end{abstract}
\begin{document}

\flushbottom
\maketitle

\thispagestyle{empty}

\section*{Introduction}

Digital technology is causing a sea-change in all parts of the medical
profession. In particular the meteoric rise of AI in general and deep
learning in particular raises the possibility that doctors will be
replaced computers~\cite{Mukherjee2017}. The father of deep learning,
Geoff Hinton, said in 2017: "It's just completely obvious that that in
ten years deep learning is going to do better than Radiologists
... They should stop training radiologists now".

Other deep learning researchers provide a more nuanced
perspective. Sebastian
Thrun~\cite{Mukherjee2017,esteva2017dermatologist} argues that
"... deep learning devices will not replace dermatologists and
radiologists. They will {\em augment} professionals, offering the
expertise and assistance".

Using computers to augment human intelligence rather replace it, is,
at the same time, both heady and boring. On the heady side, consider
cyborgs whose anatomy is part human, part artificial and can with
equal ease solve complex equations or write poetry. On the mundane
side, think of smartphones that are quickly becoming an inseparable
part of our person.
 
The idea of using computers to augment or amplify human intelligence
has a very long history. The acronyms AI (Artificial intelligence) and
IA (Intelligence Amplification or Intelligence Augmentation) have both
become popular in the early
1960's\cite{ashby1957introduction,engelbart1962augmenting}. These
days, the acronym AI is popular, while the acronym IA is not. However,
Sebastian Thrun's statement indicates that the idea of Intelligence
augmentation is still on people's mind. We suggest bringing it back.

{\bf What would IA look like when applied to medicine?} We argue that
one important ingredient is to endow AI agents with a degree of
humility. Specifically, to allow classifiers, such as DNNs, to say "I
don't know".

\section*{Uncertainty in medicine}

In machine learning, each example has a {\em label} which defines the
``ground truth'' for the example. In~\cite{esteva2017dermatologist}


Doctors and diagnostician operate in a space of high uncertainty.

\paragraph*{The diagnostic process of elimination}
\paragraph*{Data Quality, Calibration, resolution}
Discuss issue as placement of sensors, lighting when analyzing skin
lesions. Sensing back for re-testing.
\paragraph*{Protocols}

\section*{Uncertainty in Machine Learning}
Committees, Agreement, Easy and Hard cases
Our approach for distinguishing easy and hard cases.

Especially with very large data: ecg for 14 says....

\paragraph*{The semantics of ``I don't know''}
Based on conforming / contradictory experience. Not on conditional
probability.


\section*{Agency and Augmentation}
\paragraph*{Computer aided diagnostics}
Especially with very large data: ecg for 14 says....

Pathology.

\paragraph*{Dissemination of expertise}
Computers, trained by experts, can help novices.  Serves a function
similar to score-cards.

Teaching young diagnostics
\paragraph*{Confidence, Trust and adoption of technology}

\section*{Summary}

\bibliographystyle{alpha} 

\bibliography{medbib}

\end{document}