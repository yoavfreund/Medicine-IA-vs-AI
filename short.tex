\documentclass[11pt]{pnas-new}
% \documentclass[10pt]{article}
\templatetype{pnasresearcharticle} % Choose template 
% {pnasresearcharticle} = Template for a two-column research article
% {pnasmathematics} %= Template for a one-column mathematics article
% {pnasinvited} %= Template for a PNAS invited submission

\usepackage[utf8]{inputenc}
\usepackage[T1]{fontenc}
%\usepackage{xcolor,ulem}
\usepackage{mdframed,ulem}
\usepackage{wrapfig}
\usepackage{multicol}
\setlength{\columnsep}{1cm}
\usepackage{graphicx}
\usepackage{adjustbox}
\usepackage{lipsum}
\newlength{\strutheight}
%\usepackage{soul} % for strike-through (\st)

\author[1]{Hau-Tieng Wu}
\author[2]{Yoav Freund}

\affil[1]{Duke, Mathematics and Statistical Science, Durham, 27708, USA}
\affil[2]{UCSD, Computer Science, San Diego, 92093, United States. yfreund@eng.ucsd.edu}


\title{AI devices should tell you when they don't know}

\newcommand{\comment}[3]{{\color{#1} {\bf #2 :} #3}}
%\newcommand{\comment}[3]{}  % suppress comments
\newcommand{\hautieng}[1]{\comment{blue}{Hautieng}{#1}}
\newcommand{\yoav}[1]{\comment{red}{Yoav}{#1}}

\mdfsetup{middlelinecolor=blue, middlelinewidth=2pt, backgroundcolor=blue!10, roundcorner=10pt}

\mdfdefinestyle{Medicine}{middlelinecolor=red, middlelinewidth=2pt, backgroundcolor=red!10, roundcorner=10pt}
\mdfdefinestyle{ML}{middlelinecolor=green, middlelinewidth=2pt, backgroundcolor=green!10, roundcorner=10pt}
\mdfdefinestyle{Org}{middlelinecolor=blue, middlelinewidth=2pt, backgroundcolor=blue!10, roundcorner=10pt}

\newcommand{\block}[3]{
  \begin{wrapfigure}{r}[34pt]{-10pt}
    \begin{minipage}[t]{12cm}
      \begin{mdframed}[style=#1]{\footnotesize{\bf #2}\\ #3} \end{mdframed}
    \end{minipage}
  \end{wrapfigure}
}
\newcommand{\Medicine}[2]{\block{Medicine}{#1}{#2}}
\newcommand{\ML}[2]{\block{ML}{#1}{#2}}
\newcommand{\Org}[2]{\block{Org}{#1}{#2}}



\begin{document}
\maketitle

The idea that AI will soon revolutionize medicine has a long history.
In 1959, Ledley and Lusted published an article in the journal
Science~\cite{ledley1959reasoning} with the title ``Reasoning
foundations of medical diagnosis'' and the subtitle ``Symbolic logic,
probability and value theory aid our understanding of how physicians
reason''. Note, while the methods proposed are different from the deep
learning methods (DL) popular today, the goal is familiar: create a
model of human thought process and replace the doctor with a computer.

Little progress has been made towards this goal in the 60 years since
the publication of the Ledley and Lusted paper. It makes one wonder
whether better progress can be expected for the next 60
years. Granted, deep learning is a powerful new tool. However, the
problem might be with the goal, not with the tools used to achieve it.

Let us focus on diagnosis, the goal of diagnostic AI is to develop an
algorithm that produces a diagnosis that, when compared with ground
truth, has equal or better accuracy than human diagnosticians. Several
studies have claimed success in that regard~\cite{}. One controversy
regarding these papers regards the definition of ground truth. In many
cases there is no objective way to determine the correct diagnosis and
the fall back solution is to use the concensus of board certified
diagnosticians, a solution with many problems of its own.


\section{References}
% \bibliographystyle{alpha} 
\bibliography{medbib}

\end{document}

