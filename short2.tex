\documentclass[11pt]{pnas-new}
% \documentclass[10pt]{article}
\templatetype{pnasresearcharticle} % Choose template 
% {pnasresearcharticle} = Template for a two-column research article
% {pnasmathematics} %= Template for a one-column mathematics article
% {pnasinvited} %= Template for a PNAS invited submission

\usepackage[utf8]{inputenc}
\usepackage[T1]{fontenc}
%\usepackage{xcolor,ulem}
\usepackage{mdframed,ulem}
\usepackage{wrapfig}
\usepackage{multicol}
\setlength{\columnsep}{1cm}
\usepackage{graphicx}
\usepackage{adjustbox}
\usepackage{lipsum}
\newlength{\strutheight}
%\usepackage{soul} % for strike-through (\st)

\author[1]{Hau-Tieng Wu}
\author[2]{Yoav Freund}

\affil[1]{Duke, Mathematics and Statistical Science, Durham, 27708, USA}
\affil[2]{UCSD, Computer Science, San Diego, 92093, United States. yfreund@eng.ucsd.edu}


\title{AI as a team player}

\newcommand{\comment}[3]{{\color{#1} {\bf #2 :} #3}}
%\newcommand{\comment}[3]{}  % suppress comments
\newcommand{\hautieng}[1]{\comment{blue}{hautieng}{#1}}
\newcommand{\yoav}[1]{\comment{red}{Yoav}{#1}}

\mdfsetup{middlelinecolor=blue, middlelinewidth=2pt, backgroundcolor=blue!10, roundcorner=10pt}

\mdfdefinestyle{Medicine}{middlelinecolor=red, middlelinewidth=2pt, backgroundcolor=red!10, roundcorner=10pt}
\mdfdefinestyle{ML}{middlelinecolor=green, middlelinewidth=2pt, backgroundcolor=green!10, roundcorner=10pt}
\mdfdefinestyle{Org}{middlelinecolor=blue, middlelinewidth=2pt, backgroundcolor=blue!10, roundcorner=10pt}

\newcommand{\block}[3]{\begin{mdframed}[style=#1]{\bf #2}\\ #3 \end{mdframed}}
\newcommand{\Medicine}[2]{\block{Medicine}{#1}{#2}}
\newcommand{\ML}[2]{\block{ML}{#1}{#2}}
\newcommand{\Org}[2]{\block{Org}{#1}{#2}}

%% \Medicine, \ML and \Org generate different colored boxes for short
%% explanations in different areas.
%% The command format is \ML{Title}{text}
%%
%% Example:
%%\ML{Title}{
%%  bla bla bla  bla bla bla bla bla bla bla bla bla
%%  bla bla bla  bla bla bla bla bla bla bla bla bla
%%}


\begin{document}
\maketitle

The idea that AI will soon revolutionize medicine has a long history.
In 1959, Ledley and Lusted published an article in the journal
Science~\cite{ledley1959reasoning} with the title ``Reasoning
foundations of medical diagnosis'' and the subtitle ``Symbolic logic,
probability and value theory aid our understanding of how physicians
reason''. Note, while the methods proposed are different from the deep
learning methods (DL) popular today, the goal is familiar: create a
model of human thought process and replace the doctor with a computer.

Little progress has been made towards this goal in the 60 years since
the publication of the Ledley and Lusted paper. It makes one wonder
whether better progress can be expected for the next 60
years. Granted, deep learning is a powerful new tool. However, the
problem might be with framing of the goal, not with the technology
used to achieve it.

We focus on diagnosis, the goal of diagnostic AI is to develop an
algorithm that produces a diagnosis that, when compared with ground
truth, has equal or better accuracy than human diagnosticians. Several
studies have claimed success in that regard~\cite{}. Aside the
question of whether or not the claims are true, one might ask whether
comparison to ground truth is the relevant performance measure.

Imagine a doctor of oncology, Dr. Ma, that is diagnosing patients in
her clinic.  For most patients Dr. Ma gives a confident diagnosis
after a brief checkup. Other onclogists, had they been asked, are
likely to agree with Dr. Ma on this diagnosis. This does not imply
that the diagnosis is correct. Indeed it is possible that Dr. Ma and
all of the other oncologists are incorrect.

In some cases Dr Ma will not be certain about the patient diagnosis
the patient after a short visit. In such cases the doctor needs to gather
additional information. The additional information can take many
forms: lab tests, genetic tests, articles or books, consultation with
other doctors etc.  As Dr. Ma accumulates information, she looks for a
diagnostic that is consistent with the information. At some point
most of the accumulated information points towards a particular
diagnosis, which Dr. Ma uses to choose a treatment plan.

Note that in both cases the final diagnosis is based on agreement. In
the first case, it is an agreement among a putative set of alternative
doctors, in the the second case, it is an agreement among the sources of
additional information. The reason that Dr. Ma can be confident in her 
diagnosis is because it is based on many sources of information. The
confidence is based on the preponderance of evidence, rather than on a
single logical argument.  

Another aspect of this decision process is that, at the end of the
day, it is Dr. Ma that decides on the treatment plan. By doing so she
uses her own experience to decide which information to follow and
which to ignore. The treatment decision is different from the
diagnostic decision in that it is responsible for the final
outcome. As such it requires that Dr. Ma take into account aspects
outside diagnostic information such as finance, age, quality of life,
and other ethical issue.


{\em What computers can do in this context}

\section{References}
% \bibliographystyle{alpha} 
\bibliography{medbib}

\end{document}

